\subsubsection*{Purpose, Resolution Requirements, Description, Mass, Channel Count}
The main physics program for the GlueX experiment will be conducted with a
low-power liquid hydrogen target. The planned target is $30\, cm$ long and
somewhere between $3$ and $6\, cm$ in diameter. Such targets normally 
employ mylar target cells. The mylar cell will be mounted on a metal base 
to provide for liquid entry ports and a reliable means of positioning the 
cell. The beam enters through a thin window mounted on a reentrant tube at 
the base of the cell. The target cell is connected to a condenser located 
upstream of the cell.

\subsubsection*{R\&D Issues, Simulations, Monitoring and Other Considerations}
The maximum power deposited in the target by the beam is $100\, mW$.  In such low-power
targets, natural convection is sufficient to remove heat from the target cell and a
circulation pump is not required. A system such as this, containing a few hundred 
$cm^3$ of liquid hydrogen, would be considered ``small" by Jefferson laboratory standards
and the safety requirements would not place any significant constraints on the target
design or operation.

\subsubsection*{Manpower, R\&D and Production Schedules}
The target will be built by the JLab target group. It is estimated that
approximately one year will be required to design, construct, test and
certify the target. This target is very similar to other targets that
have been built at JLab, partcularly for Hall B.


\subsubsection*{Purpose, Resolution Requirements, Description, Mass, Channel Count}
The start counter is used to provide a start signal for
time of flight measurements and to identify  the beam pulse associated
with the observed event. In order to be independent of particle momenta
and trajectories, the start counter is located as close to the target as
possible. To be able to identify beam pulses the detector
needs a minimal time resolution of 300~ps. 

The detector will consist of an array of 30 to 60 scintillators
cylindrically surrounding the target. The inner diameter of the
cylinder is 10~cm. Each scintillator element will have  
a thickness of 3 to 5~mm and a length between 45~cm and
70~cm. The optimal dimension will be determined from simulation. The
detectors can be read out either via high magnetic field PMT's or
other detection methods based on solid state detectors.     

\subsubsection*{Raw Signals, Stages of Amplification, Final Readout}
>From studies using cosmic rays we have determined that high field
PMT's such as the Hamamatsu R5942 (H6614-01 system) can provide a time
resolution of 250~ps which makes is suitable for our application. For
minimum ionizing particles we obtained signal sizes of about 0.4~V
with a rise time of about 5-8~ns.

\subsubsection*{R\&D Issues, Simulations, Monitoring and Other Considerations}
The timing performance of the Hamamatsu tube his to be determined in
various magnetic field configurations. Other readout methods including
VLPC and SiPM will also be studied to investigate the feasibility of a
double ended readout system. Further simulations are needed to finalize
the detector geometry.

\subsubsection*{Manpower, R\&D and Production Schedules}
FIU is responsible for the start counter. In its current form without
the need of high resolution position information the available man
power at FIU is sufficient. Once R\&D work using VLPCs is completed,
we expect that the counter can be built on the time scale of two years.




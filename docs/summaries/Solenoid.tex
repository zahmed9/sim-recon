\subsubsection*{Purpose and Requirements}

The Solenoid is the magnetic element selected for the GLueX Experiment to
provide momentum analysis in the tracking chambers. The Solenoid
is a 73-inch warm bore super conducting (SC) device that produces
a nominal maximum central field of 2.2 Tesla at 1800 Amps. The Magnet is
195 inches long and weighs  approximately 300 tons. The Solenoid was originally
 designed in 1970 as the LASS spectrometer at
SLAC and was subsequently used as the MEGA spectrometer at LANL. The
Solenoid was mothballed in place
in 1995 and first inspected by JLAB Hall D in 2000. The solenoid was
designed as a highly reliable cryogenically stablized SC magnet.
The typical field quality of the solenoid (dB/B $\approx$ 1 \%) is consistent with GlueX
requirements for momentum resolution. The Solenoid was
designed with a slotted yoke and four separate cryostats to accomodate 
the large planar wire chambers used at LASS.
The specific magnetic geometry of the original solenoid had the unfortunate 
side effect of
creating large external magnetic fields, particularly in regions where
phototubes would likely be present. Many of the Solenoid's systems were
inconsistent with JLab operations due either to design or obsolescence; 
the Solenoid had a history
of internal leaks and electrical shorts and had accumulated a substantial 
amount of wear and tear during its 30 year history.
Despite these problems, the robust design and the good state of
preservation made the selection of this solenoid a cost effective
choice for Hall D, even after the necessary costs of modernization and
maintenance were taken into consideration.

\subsubsection*{Status of Upgrading, Modernization and Repair.}

 The first task
performed after the initial evaluation of the Solenoid was to redesign the
yoke magnetic geometry to match the Solenoid to the requirements of Hall D
and to reduce substantially the external fields. The proposed modifications
entail filling the yoke slots and adding steel at the end to reduce yoke
saturation and the escape of internal fields, thus reducing the external
fields by creating a symmetric yoke with a large upstream opening. This
modification also changed the solenoid to a clear bore from end to end.

The MEGA Solenoid was dismantled and shipped to IUCF where it has been 
undergoing some long needed maintenance and repairs. Two of the four
coils are now finished. Refurbishment included repairing LN2 leaks in three coils and a leak
in the LHe vessel of the forth. The shield thermometry is being upgraded to
modern PT-102 Pt resistance thermometers, and the coil support
strain gauges and the deteriorated multi layer super insulation are all being replaced.
The two completed coils were cooled down with
LN2 in July 2004 to perform a final cold leak test and they were
determined to be leak-free to the highest industry standards. This cold testing also 
confirmed that the new instrumentation and insulation worked as designed. 

A mid-course assessment of the solenoid by external experts was held in December of 2004 to
provide advice and recomendations on the present course of restoration and future 
plans for the solenoid. The most significant recomendation was to consider actions 
to eliminate the coil shorts. This advice was taken to heart and a plan was 
concieved to perform a surgical short repair on coil 3 which has a  
hard short to ground. This extra work was incorporated into the work plan for coil 3,
and work on coils 3 and 4 is due to be completed by March 2006.

New Solenoid systems are planned that include a new DC power
supply and energy dump system recently purchased from Danfysik for GlueX.
A new cryogenic interface with JLab compatible automatic valves and
connections and a new control and instrumentation package is under design at JLab.

\subsubsection*{Manpower, R\&D and Production Schedules}

The refurbishment of the coils is being performed at IUCF under
the direction of JLab staff, and will continue until complete.
A cost-benefit analysis is underway to determine the ultimate value 
of a full-scale test of the solenoid including the iron yoke 
in advance of the availability of Hall D, possibly at IUCF.
Ultimately the solenod will be installed and thoroughly tested in Hall D 
prior to the installation of the actual GLUEX apparatus. This installation
is one of the schedule drivers for the experiment, and therefore must be 
streamlined to allow prompt assembly of the detector.
\subsubsection*{Purpose, Resolution Requirements, Description, Mass, Channel count}

The purpose of the CDC is to accurately measure $(r,\phi,z)$
coordinates along charged-particle tracks. In conjunction with the FDC, it
will then  reconstruct the momentum, $\vec{p}$ of each track and the primary and
secondary vertices's of the event. The exact momentum resolution is a function 
of particle momentum and the number of hits in both the CDC and FDC. Monte Carlo
studies indicate that an $r\phi$-spatial resolution, $\sigma_{r\phi}$, on the 
order of $150$ to $200\mu m$ is sufficient to satisfy the physics goals of the 
experiment. The $z$-coordinate is obtained using $6^{\circ}$ stereo layers. 
The resolution is given as $\sigma_{z}=\sigma_{r\phi}/\sin 6^{\circ}$.  
The CDC also needs to provide $dE/dx$ information sufficient for separating 
$K$s and $\pi$s for particle momentum under $0.500\,GeV/c$.

The chamber is built using 23 layers of $1.6\,cm$ diameter, $100\mu$ thick
aluminized kapton, $2\,m$ long straw tubes. The signal is measured using a
$20\mu$ diameter gold-plated tungsten wire strung under $55\, g$ of tension.
Layers $5$, $6$, $14$ and $15$ are $+6^{\circ}$ stereo while layers
$7$, $8$, $16$ and $17$ are $-6^{\circ}$ stereo. Due to the packing of the
tubes, the exact channel count depends on the exact specifications of the
final chamber, but is presently estimated to be $3240$ channels.

\subsubsection*{Raw Signals, Stages of Amplification, Final readout}

A minimum ionizing track produces about $30$ primary ionizations per 
centimeter of traversed gas. Path length in a straw-tube depends on
both the distance away from the wire as well as the polar-angle $\theta$
of the track, but typical values vary between about $0.5\, cm$ to a few $cm$.
The chamber will be run such that the gas amplification is about $10^{4}$.
The signals will be read out using capacitively coupled preamps mounted
directly on the upstream end plate of the detector and then fed into 10+ bit
FADC and digitized at 125~MHz. Due to the $2.25\, T$ magnetic field, the maximum
drift times will be on the order of $800~ns$ for a typical gas mixture. 


\subsubsection*{R\&D Issues, Simulations, Monitoring and Other Considerations}

The group is currently stringing wires in a $\frac{1}{4}$-chamber, full-scale 
prototype. We have currently identified several design changes that will
facilitate easier construction. Apart from construction technique, the main
 issues to be resolved with the prototype are gas distribution and electronic
hook-ups.  


\subsubsection*{Manpower, R\&D and Production Schedules}

The CDC is the responsibility of the Carnegie Mellon University group. Assuming that
a team of stringers is hired during the actual fabrication phase (as has been
done with other chamber projects), the group has sufficient manpower to build the
final device on a time scale of three and a half years from the time that funds
become available. The group expects to work with the FDC team and the JLab electronics
group to build a preamp that is common to all chambers in the experiment.  


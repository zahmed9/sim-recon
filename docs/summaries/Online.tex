\subsubsection*{Purpose, Description}

The online effort is the umbrella under which all efforts related to
taking data and writing it to mass storage will be organized, and
includes overall responsibility for designing, installing, and
maintaining everything related to controlling and running the
experiment.  This effort covers
\begin{itemize}
\item electronics and trigger installation
\item experiment network design and installation
\item counting house and operator environment
\item DAQ installation and customization
\item run management and control
\item construction and maintenance of level 3 farm
\item online event monitoring 
\item online event display
\item electronic logs and bookkeeping
\item slow controls
\item alarm systems
\item online calibration farm
\end{itemize}


\subsubsection*{Current Status, R\&D issues}

Most of the R\&D needed here is being done by other groups, especially
the Electronics, DAQ, and Trigger groups.  For the remainder we will
adapt existing or soon to exist technology (from JLab, CERN, Fermilab,
SLAC, industry, etc.) as needed.  The main effort will go into
customizing the chosen technologies and integrating them into a
coherent and usable online system.



\subsubsection*{Manpower, Further R\&D, Production}

GlueX will take primary responsibility for the online effort, and we
expect that a large fraction of the work will be done by experimenters
residing at JLab.  The online software effort is substantial,
involving customization and integration of a wide variety of outside
packages.  The online hardware effort is also substantial, involving
integration of a large number and wide variety of electronics,
computing, and monitoring equipment.  Careful planning for dedicated
manpower is necessary for successful implementation of the online
effort.

Much of the online hardware and software needs to be available two to
three years before the beginning of data taking to allow for detector
and electronics commissioning.

\subsubsection*{Purpose, Resolution Requirements, Description, Mass, Channel count}
The purpose of the Tagger system is to provide a flux of $\sim 10^{8}$~Hz
of linearly polarized photons from coherent bremsstrahlung in a thin,
orientated diamond crystal. This is achieved by measuring the energies
of the energy degraded bremsstrahlung electrons in the spectrometer.
The photon energy resolution is required to be less than 0.1\% r.m.s.\
of $E_{0}$ for $E_{\gamma}$ between 70\% and 75\%  of $E_{0}$, which
corresponds to 12~$MeV$ r.m.s. energy resolution for a 12~$GeV$ electron beam. 
The Tagger system consists of a quadrupole and 2 dipole magnets, 
a vacuum chamber and the associated focal plane detectors. The dipoles
are two identical magnets that will be run at 1.5~T.  According to the
present design, the pole shoe surfaces will be part of the vacuum chamber.
The focal plane detector array is located just outside the vacuum chamber.
It consists of a set of 141 fixed scintillation counters spanning the full
energy range from 25\% to 95\% of $E_{0}$ and is required for the alignment
of the diamond. A movable ``microscope'' of 120 narrow channels is required
to accurately measure the photon energies in the energy  range 70 to 75\%
of $E_{0}$.  The total mass of the tagging spectrometer system will be
$\sim$ 90 tons.  

\subsubsection*{Raw Signals, Stages of Amplification, Final readout}

Since individual detectors in the focal plane array will have to count
at rates up to $5\times 10^{6}$~Hz, a plastic scintillator/photomultiplier
combination is appropriate.  The fixed scintillator array consists of
paddles 0.5~cm thick mounted perpendicular to the electron trajectories
which span 0.5\% $E_0$ each in recoil electron energy for a total of 141
counters.  The microscope consists of a rectangular array of square
scintillating fibers connected to clear plastic light guides.  Each
fiber is read out with a silicon photomultiplier device, which provides
excellent gain, rate capability and time resolution.  The signals from
the individual fibers are ganged together to form 120 independent tagger
channels of approximately 0.1~\% $E_0$ each.  The digitization electronics
from the focal plane detectors will be standard.


\subsubsection*{R\&D Issues, Simulations and Other Considerations}
The Glasgow  and Catholic Universities groups have calculated the tagger
optics separately, the results of which can be found in Hall D
note 70\footnote{Optics calculation for the Hall D tagging spectrometer
for a 12 GeV electron beam. G. Yang, January 2004.} and the GlueX/Hall D
Design Report (Nov 2002). The Design Report tagger is different from
the design described above. It has a single long, narrow dipole which is
6.1 m in length and weighs $\sim$ 100 tons. Due to concerns about the
mechanical stiffness, the availability of sufficiently large pieces of
iron of the necessary quality and the availability of suitable
manufacturers, Glasgow  and Jlab investigated the possibility of a
tagger consisting of two identical magnets in series\footnote{Optics
calculation for a two magnet tagged photon spectrometer for GlueX.
G. Yang, July 2004.}.
By  careful positioning of the two magnets it is possible to obtain a
design that is equivalent, and in some respects superior, to a single
magnet configuration. The two magnet design concept was accepted by
the collaboration at the Indiana meeting in May 2004. Glasgow has studied
the design in more detail and has produced drawings of the two magnet
assembly, including a possible vacuum system, which should contain
sufficient details for budget prices to be obtained from potential
manufacturers.
It is also relevant to mention that prior to version 4 of the Design Report, 
Glasgow and Jlab investigated the feasibility of a tagger with superconducting 
coils\footnote{Possible designs for a 12 GeV superconducting Tagger. J. Kellie, 
November 2001.} with a magnetic field of 5T and main beam bend angles of 15, 30 
and 45 degrees, for both curved and straight output edges - the room temperature 
tagger bend is 13.4 degrees. After careful consideration, the superconducting 
option was rejected since there are several distinct disadvantages and no 
clear advantages.

\subsubsection*{Collaboration Responsibilities}

The Tagger system has been the responsibility of groups from Glasgow 
University, JLab  and Catholic and Connecticut Universities.  More 
work is required to investigate alternative vacuum system designs,
of which several have already been considered. 
%-we have already considered a vacuum chamber which, (i) is external 
%to the tagger dipole magnets, or (ii) uses the pole shoe surfaces as 
%an integral part of the chamber. Vacuum systems which are either 
%completely welded or use a combination of O-ring seals and welds 
%have also been examined.  
It should be realistic to obtain cost 
estimates for the magnets and vacuum system in the near future. 
The Moscow group using ISTC financing could provide the necessary 
manufacturing skill and manpower to produce the magnets and the 
vacuum chamber.  Basic R\&D is required for the focal plane 
assembly, and a decision on which group or groups should take 
on this responsibility should be made in the near future, 
bearing in mind manpower requirements.

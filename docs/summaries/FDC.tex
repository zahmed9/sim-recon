\subsubsection*{Purpose, Resolution Requirements, Description, Mass, Channel Count}

The \fdc{}s include 4 separate packages of disk-shaped horizontal drift chambers 
to measure the momenta of all charged particles emerging from the target at 
angles of up to 30$^{\circ}$ relative to the photon beam line.  Each package
consists of 6 planes of alternating anode and field-shaping wires with a 
wire-to-wire separation of 5~mm (119 anode wires per plane) and with 150~$\mu$m 
spatial resolution from the drift time readout.  Each wire plane is sandwiched 
between 2 planes of cathode strips (238 strips per plane with 5~mm pitch).  By 
charge interpolation of the electron avalanche image charge in the cathode strip 
readout, spatial resolutions at the cathode planes are expected of better than 
150~$\mu$m.  The strips are arranged in a $U$ and $V$ geometry with respect to 
the wires (at $\pm$45$^{\circ}$) allowing for separation and assignment of 
multiple hits within a chamber to the different tracks.  Adjacent chamber 
elements will be rotated by 60$^{\circ}$ with respect to each other in order 
to improve track reconstruction decisions on the corresponding anode wire 
left/right ambiguities, hence improving the overall resolution.  The wires 
that cross through the beam line will be deadened out to a radius of 3.5~cm 
to reduce the rates.  Each \fdc{} package has a channel count of 3570, leading 
to a total channel count for the full \fdc{} system of 14280 (2856 anodes and
11424 cathodes).

\subsubsection*{Raw Signals, Stages of Amplification, Final Readout}

Each signal from the \fdc{}s (anodes and cathodes) will be sent to a
chamber-mounted charge-sensitive preamplifier that drives a pulse-shaping 
amplifier. The signals from the anode wires that are above some 
pre-determined voltage threshold will be discriminated and then digitized 
by 125~ps LSB resolution F1 TDCs.  The signals from the cathodes will be digitized with 
62.5~MHz 10-bit flash ADCs.

\subsubsection*{R\&D Issues, Simulations, Monitoring and Other Considerations}

The primary development issues that must be addressed for the \fdc{} system
are factors affecting the intrinsic resolution of the chambers, along with 
the mechanical and electronics layout.  The goal is to construct a tracking 
detector that meets the required design specifications and has a long life 
time, a uniform and predictable response, a high efficiency, and is 
serviceable in case of component failure.

Two detector prototypes will be completed and studied over the course
of the next three years.  The first will be employed to study the optimal 
electrode configuration for the system.  A second full-scale prototype
will be completed to test mechanical support designs for the chamber 
cathode planes and wire planes, which is necessary to avoid electrostatic 
instabilities and non-uniformities that are known to affect resolution. 
This second prototype will also be essential to complete the final design 
of the \fdc{} circuit boards.  A significant aspect of the design work includes 
development and study of Monte Carlo of the GlueX detector system focussing 
on the properties of the \fdc{} system that will enable us to meet or exceed 
the required design specifications.

The detector group at Jefferson Laboratory is developing the gas system
for the entire GlueX experiment.  The Ohio University group will work
to ensure that this design is adequate for the control and monitoring
of the \fdc{} system.

\subsubsection*{Manpower, R\&D and Production Schedules}

The \fdc{} prototyping and design is primarily the responsibility of
the Ohio University group, with important support from the detector
group at Jefferson Laboratory.  The manpower available is adequate
to complete the detector R\&D within 3 years and to complete the
detector construction with 4 years pending availability of funds.


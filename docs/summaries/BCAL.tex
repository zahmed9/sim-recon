\subsubsection*{Purpose, Resolution Requirements, Description, Mass, Channel count}
The purpose of the BCAL is the detection and energy determination
of photons and charged particles from the decays of the neutral
$\pi$, the $\eta$ and other mesons decaying into photons.  All
charged particles that fall within its volume as they are swept
by the magnetic field, mostly in the momentum range of $300-1000\, MeV/c$,
will also be detected.  Some spatial information can also be extracted
from the timing information relative to the two read-out ends of the BCAL.
The design of the BCAL is based on that of the KLOE calorimeter at LNF
in Italy. The expected energy resolution for photons is
$\sigma(E)/E\approx 0.02+0.05/\sqrt{E}$, while the expected timing
resolution is $\sigma(t)\approx 200\,ps$.  The physical layout of the
BCAL is a ring consisting of $48$ modules (segments) at an inner radius
of $65\,cm$ and an outer radius of $90\, cm$.   Thus, its  approximate thickness
is $25\, cm$ corresponding to approximately $18-19$ radiation lengths. The nominal
length of each module is $400\, cm$.  Each module is constructed as a matrix
of $96$ double clad scintillating fiber optic strands (SciFi+IBk-s), embedded
on grooved Pb sheets of $0.5\, mm$ thickness. Thus, each module consists of
approximately $220$ layers of Pb/SciFi and special optical epoxy composite.
The high magnetic field and limited space available for read out, is an
area of particular concern with several emerging technologies, such
as SiPM's, offering the most attractive solutions. Extensive R\&D is still
needed to finalize the type of readout devices and the required channels.


\subsubsection*{Raw Signals, Stages of Amplification, Final readout}
It is clear that a large number of SciFi+IBk-s will be read out by any of the
selected PM's.  The exact number depends on PM window size and its saturation
properties.   Assuming $1\, mm^{2}$ to $3\, mm^{2}$ SiPM window area and a 
$5\,cm \times 5\, cm$ matrix area viewed, approximately $1600$ SciFi's will 
be viewed by a number of SiPM's. The optimum number of SiPM's required will 
depend on the nature and geometry
of the light concentrator and diffuser and the light collection efficiency of
the optical fibers that will transport the light to the SiPM's.   This
requires R\&D and testing of various configurations, however early studies
indicate between a number 10 and 20 SiPM's.  Each group will be coupled to
provide one (fast) analog signal similar to that from vacuum PMT's, which
are thus subject to standard ADC and TDC processing.




\subsubsection*{R\&D Issues, Simulations, Monitoring and Other Considerations}
The R\&D on the actual construction of the BCAL is almost complete with
a full-scale $4$~m module built, and tested with cosmic-rays.  The
read-out requires significant R\&D and MC simulations, more so because the SiPM
technology is still not yet mature and specifications and performance improve
with demand and experience.  This R\&D has already started and most likely will
result in collaboration with DESY, CERN, KLOE and Russian groups to custom
tailor the devices to our needs.  The monitoring system of the read-out devices
can be done by using a pulsed laser-fiber optic combination.



\subsubsection*{Collaboration Responsibilities}

The BCAL is the responsibility of the UofR SPARRO group.   For the construction
of Module 1 and subsequent production modules, as well as radiation damage
studies of the SciFi's, the CSR group at the University of Alberta is also
assisting.  This combined manpower is adequate to complete the R\&D phase
by the end of 2005.   It is also adequate to complete the construction of
the BCAL subject to external funding and adequate construction timelines.
The high-energy physics group at the University of Athens will be assisting
with the SiPM R\&D, as the European component of this emerging technology effort.


\subsubsection*{Purpose, Resolution Requirements, Description, Mass, Channel Count}

The purpose of the TOF is to serve as part of the particle identification system
in conjunction with a Cerenkov detector for forward-going charged particles.  The
goal is to separate $\pi^{\pm}$ from $K^{\pm}$ for momenta up to 2~GeV/$c$ and 
for the given geometry a 95\% separation efficiency is achieved 
at the highest momentum with a time resolution of 80~ps.  The TOF will use
two planes of scintillator bars located immediately upstream of the lead glass
detector (LGD).  Based on simulations and prototype studies the bars will be
250~cm long, 6~cm wide and 1.5~cm thick.  Thus the mass presented by the 
detector immediately before the LGD corresponds to 3~cm of scintillating plastic.
Each bar is read out at both ends with a photomultiplier.  The channel count is 168.

\subsubsection*{Raw Signals, Stages of Amplification, Final Readout}

Prototype studies with a cosmic ray test facility at Indiana U and extensive tests
in a hadron beam at IHEP (Protvino, Russia) included various scintillating bars
of different thickness (1.5~cm, 2.5~cm and 5.0~cm) with various photomultipliers
(Russian FEU-115, Hamamatsu R5506 and R5946, and Philips XP2020).  The
XP2020 was chosen.   The typical pulse has a rise-time of less than 5~ns and
an amplitude of about 0.5~V.  Constant fraction discriminators will be used and 
a TDC with a least count of 25~ps.

\subsubsection*{R\&D Issues, Simulations, Monitoring and Other Considerations}

The performance of the prototype TOF in
a 5~GeV/$c$ hadron beam at IHEP has been described in three NIM 
publications\footnote{Nucl. Instr. \& Meth. \textbf{A478} 440 (2002); 
 Nucl. Instr. \& Meth. \textbf{A494} 495 (2002);
Nucl. Instr. \& Meth. \textbf{A525} 183 (2004) }.  For the 1.5~cm thick TOF bar,
a time resolution for a two-bar system of 77~ps at the bar center and 40~ps
near the ends was achieved.  Magnetic shielding studies were also carried out
and are described in a NIM publication\footnote{Studies of magnetic shielding
for phototubes; accepted and
available online at www.sciencedirect.com 10 Aug 2004}.  Further R\&D measurements
with an array of scintillator bars in a hadron beam at IHEP are planned within the
next year.


\subsubsection*{Manpower, R\&D and Production Schedules}

The TOF is the responsibility of the groups from Indiana University and the Institute for
High Energy Physics (IHEP) in Protvino, Russia.  This manpower is adequate to complete
remaining R\&D in six months and to complete the detector construction in
two years from availability of funds.


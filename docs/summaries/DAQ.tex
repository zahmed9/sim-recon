\subsubsection*{Purpose, Description}

The data acquisition system must:
\begin{itemize}
\item support a deadtimeless front-end system at a 200 kHz trigger rate
\item collect data from the front-end modules at 1 GB/s
\item build event fragments into a single event record
\item pass built events to a level 3 farm
\item write level 3 accepted events to mass storage at 100 MB/sec
\item deliver a small subset of the events to calibration and
  monitoring systems
\end{itemize}

GlueX expects to fill approximately 70 readout crates 
containing 25,000 channels of electronics. 
We expect a level 3 reduction of 90\%, which reduces the 1 GB/s raw
rate to an accepted event data rate of 100 MB/s.

\subsubsection*{Current Status, R\&D issues}

GlueX will use CODA, the standard DAQ system developed at JLab, but at
higher trigger and data rates than have been achieved so far.  We note
that the high rate makes it impossible to interrupt the front-end
processors for every event, parallel event building will be required,
and that CODA has not been run with a high-rate level 3 farm.

CODA is under active development by the DAQ group at JLab.



\subsubsection*{Manpower, Further R\&D, Production}

Primary responsibility for developing CODA belongs to the DAQ group at
JLab.  This includes all the software needed to  
program and collect data from the front-end boards,
build events,
deliver and analyze events in a level 3 farm,
and write events to local mass storage.
The JLab computer center is responsible for moving the data (100
MB/s) from local mass storage to permanent storage at the central
computing facility.

Note that the DAQ group will develop the software infrastructure
required to run the level 3 farm, while GlueX will build and run the
farm, and develop the level 3 trigger algorithm.  We further note that
the DAQ group must be intimately involved with the GlueX electronics,
trigger, and online efforts.

A prototype DAQ system needs to be available two to three years before
start of data taking, and the production system must be ready one year
before the start of data taking.


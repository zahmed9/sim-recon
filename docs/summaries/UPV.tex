\subsubsection*{Purpose, Resolution Requirements, Description, Mass, Channel count}

The purpose of the UPV is to detect backward going photons of energy greater than 20 MeV 
emerging from the target region. The UPV provides the upstream coverage of the hermetic 
photon detection. The design of the UPV employs a traditional lead-scintillator sampling 
calorimetry.   The detector is able to detect multiple photons with fast detection and 
with timing information that may be utilized at the trigger level.

The UPV consists of 18 layers of scintillator alternating with first 12 layers of lead 
sheets (0.36 radiation length thick) then 6 layers of  lead sheets (0.72 radiation 
length thick). The expected energy resolution for incident photons is 
$\sigma(E)/E = 5\% + 8\%/\sqrt{E}$ at a $24\%$ sampling fraction. Each  scintillator 
layer consists of seven 34cm x 238 cm paddles forming a plane. The central paddle 
has a 10 cm hole to allow for the passage of the beam. The effective area of each 
plane is approximately 238 cm x 238 cm. The total counter thickness is 8.91 
radiation lengths.   The layers are arranged into 3 alternating orientations: 
x, u, and v ( $\pm45^o$, respectively).  


The scintillation light is collected at one end of each paddle only.  For each 
orientation, the light collecting ends of the scintillators are join together 
via a wavelength shifter which is oriented perpendicular to the  scintillators.  
The wavelength shifter is used to redirect the light through $90^o$ and out the 
upstream end of the solenoid to photomultipliers tubes (PMT).  Each PMT is 
protected from fringe magnetic field with soft steel casing and mu-metal shield. 
The channel count is 21.

\subsubsection*{Raw Signals, Stages of Amplification, Final readout}

A typical pulse has a signal of 500 mV and a rise time of 10 nSec corresponding 
to about $10^3$ photoelectrons at $10^7$ PMT gain. The signal will be digitized 
using standard FADC and TDC modules.  

\subsubsection*{R\&D Issues, Simulations, Monitoring and Other Considerations}

Initial R\&D on the construction and building techniques of a prototype module 
is near completion. Measurements of prototype performance in a low energy 
electron beam are planned for later this year.  These results will be compared 
to Geant4 simulations.   R\&D continues on optimizing light collection, 
channel segmentation, and read-out. SiPM's, an emerging technology which is 
being studied for use in the BCAL, offer an attractive read-out solution 
along with potential benefits in light collection and in minimizing high 
magnetic field effects. The monitoring system of the UPV read-out can be 
tied into the system utilized for the BCAL detector, which uses a pulsed 
laser-fiber optic system.



\subsubsection*{Manpower, R\&D and Production Schedules}

The UPV is the responsibility of the group from Florida State University.  
This manpower is adequate to complete remaining R\&D and to complete the 
detector construction in two years from availability of funds.


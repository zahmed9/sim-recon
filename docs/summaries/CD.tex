\subsubsection*{Purpose, Resolution Requirements, Description, Mass, Channel count}

The Cerenkov detector is to serve as part of the particle identification system
together with the time-of-flight system for forward-going charged particles.
The goal is to distinguish between pions, kaons and protons with
momenta above the reach of the time-of-flight system. Here we describe the DIRC,
one of options considered for this task.

The DIRC uses thin, long rectangular bars made of synthetic fused silica
(quartz) both as Cerenkov radiators and light guides  with
an index of refraction of $n \approx 1.47$. The design is based on the
successful novel technique for particle identification developed for
BABAR \footnote{I. Adam \em{et al.} Nucl. Instr. \& Meth. \textbf{A538} 281 (2005).}.
A fraction of the Cerenkov photons produced by relativistic charged particles 
in the quartz bar remain inside the bar due to total internal reflection.
These photons are collected at the end of the bars and imaged onto 
an array of photomultiplier tubes preserving sufficient angular precision to
determine the angle of the produced Cerenkov light. The DIRC can provide 
identification of pions, kaons and protons with an efficiency of better 
than 90\% and a mis-identification rate below 10\% for momenta up to 4.5 GeV/c. 
The current design for GlueX requires 60 quartz bars and 2000 photomultiplier 
tubes.

\subsubsection*{Raw Signals, Stages of Amplification, Final readout}

An electron will create about 30 detected photoelectrons in the struck bar.
The photomultipliers need to be sensitive to single photoelectrons
with sufficient time resolution to minimize out-of-time photons converting
in the quartz. The rate per tube is expected to be approximately 100kHz.
The photomultiplier tubes will be placed at a distance of
about 3 m on one side of the GlueX detector where the fringe fields of 
the solenoid are below 100 G. Efficient shielding of the field can be achieved
with standard techniques.


\subsubsection*{R\&D Issues, Simulations, Monitoring and Other Considerations}

While the principle and hardware of the BaBar DIRC with modifications of 
the geometry are applicable to the GlueX detector we explore possible 
improvements of the imaging. 
The exact exit position of a Cherenkov photon from a bar is unknown.
At BaBar this uncertainty is reduced by placing the phototubes at some
distance to measure the photon direction from the bar. This pinhole focusing
requires a large water tank. On the other hand, focusing of the photons
with mirrors onto multi-channel photon detectors increases the spatial
resolution and allows a compact assembly. Such devices also provide higher
precision in the measurement of the photon arrival time which allows to
correct for the significant smearing of the Cherenkov angle due to the 
unknown photon wavelength. As for the readout we investigate leading-edge 
versus constant-fraction discrimination together with the TDC solution of 
GlueX. A prototype of a focusing DIRC with cylindrical mirrors is currently
being tested at SLAC
\footnote{C. Field \em{et al.} Nucl. Instr. \& Meth. \textbf{A518} 565 (2004).}.
Simulation studies are performed to optimize the mirror imaging and 
the final design.

\subsubsection*{Manpower, R\&D and Production Schedules}

The Cerenkov detector for GlueX is not as well developed as other detector components.
This is due to the fact that the first group to express interest in this detector
is no longer part of the collaboration. After the granting of CD0 to the JLab upgrade,
a pair of groups from Tennessee approached the collaboration and are eager to take
on the responsibility for a major detector. Based on a fresh look at the challenges
for particle identification and their own expertise, they have proposed the
construction of a DIRC detector which is a very good match to particle identification
in this momentum range. A final decision on the detector technology we will use clearly
depends on many factors including physics, manpower, costs and timescales. The collaboration
is currently evaluating all of these. 

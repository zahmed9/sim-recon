\subsubsection*{Purpose, Resolution Requirements, Description, Mass, Channel count}

The Cerenkov detector is to serve as part of the particle identification system
together with the time-of-flight system for forward-going charged particles.
The goal is to distinguish between pions, kaons and protons with
momenta above the reach of the time-of-flight system. Two options exist for this
detector: a threshold gas option using C$_4$F$_{10}$, and a quartz-based DIRC
option. Here we describe the gas Cerenkov option.

The gas Cerenkov counter operates as a threshold detector, tagging relativistic
pions above the threshold for emission of Cerenkov light. Several radiator materials have
been considered for the design. A pressurized gas radiator has the flexibility of
matching the index of refraction to the desired momentum range, but requires
thick gas containers in the downstream detector region. Two atmospheric-pressure
radiators were found to produce high acceptance rates: aerogel (n=1.008) and C$_4$F$_{10}$
(n=1.0015). The C$_4$F$_{10}$ radiator was selected because it has a momentum
threshold for pions of 2.5 GeV/c.

The proposed optics collects light from two sets of mirrors, which focus
the light onto 40 photomultiplier tubes. All forward particles detected by the
time-of-flight system and lead glass detectors will traverse one set of 
mirrors. Therefore, there is a hight premium for producing these mirrors with
minimal thickness.


\subsubsection*{Raw Signals, Stages of Amplification, Final readout}

The performance of the system is expressed in terms of the average number
of photoelectrons detected per track. The radiator thickness is a minimum
of 80 cm of gas. A reasonably good design will produce
approximately 22 photoelectrons in the relativistic limit. For 95\%
detection efficiency, this sets the effective threshold of pions at 3 GeV/c.
The photomultiplier tubes will be placed about one meter away from the
opening of the solenoid, so special care will be required to
shield the photomultiplier tubes from the the fringe fields of the magnet.

\subsubsection*{R\&D Issues, Simulations, Monitoring and Other Considerations}

Gas-filled Cerenkov detectors have been used in many particle physics 
experiments. Still, it is a challenge to collect
sufficient light for efficient detection of relativistic
particles. Many requirements must be balanced to achieve high efficiency.
The optics must to be designed to collect light to locations
where photomultiplier tubes can be placed and operated in the fringe
fields of the magnet. At the same time, the mirrors must have high reflectivity for the UV Cerenkov
photons as well as be thin to have minimal impact on particle trajectories.
The choice of radiator is also another variable. For example, the 
original LASS spectrometer used a freon radiator in a 
similar design to GlueX, but there is a limited selection of heavy
environmentally friendly gases. 

\subsubsection*{Manpower, R\&D and Production Schedules}

The Cerenkov detector for GlueX is not as well developed as other detector components.
This is due to the fact that the first group to express interest in this detector
is no longer part of the collaboration. However, after the granting of CD0 to the JLab upgrade,
a pair of groups from Tennessee approached the collaboration and are eager to take
on the responsibility for a major detector. Based on a fresh look at the challenges
for particle identification and their own expertise, they have proposed the
construction of a DIRC detector which is a very good match to particle identification
in this momentum range. Because this proposal is so new, the collaboration is still
evaluating the relative merits of each option.

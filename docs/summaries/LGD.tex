\subsubsection*{Purpose, Resolution Requirements, Description, Mass, Channel Count}

The purpose of the LGD is to detect and measure the energy and position of photons
from the decays of $\pi^0$, $\eta$ and other mesons.  LGD's of similar construction
were used in experiments at Brookhaven (E852 - using a pion beam) and JLab (Radphi
-using a photon beam).  
 The energy resolution
given by $\sigma(E)/E = 0.036 + 0.073/\sqrt{E}$.
Shower positions at the LGD plane
 are reconstructed with a resolution
 of $\sigma_r = \sqrt{(7.1/\sqrt{E})^2 + (X_0 \sin\theta)^2}$~mm
 where $X_0$ is the radiation length of the lead glass
 (30~mm) and $\theta$ is the photon angle measured with respect
 to the normal to the LGD (energies in GeV). This leads to mass resolutions of 10~MeV/$c^2$
 and 30~MeV/$c^2$ for the $\pi^0$ and $\eta$ respectively.
   The detector consists of 2300 lead
glass blocks of dimensions $4 \times 4 \times 45$~cm$^3$ arranged in a nearly circular
stack of radius $\approx$1~m.
  The Cerenkov light from each block is viewed by a FEU-84-3 Russian phototube. 
The phototube bases are of a Cockcroft-Walton (CW) design.  The phototubes are resigtered 
with respect to the glass using a cellular wall that includes soft-iron and $\mu$-metal
shielding. Since the LGD is the furthest downstream subsystem in the overall
GlueX detector, the mass presented to particles is not an issue.  The channel count is 2300.

\subsubsection*{Raw Signals, Stages of Amplification, Final Readout}

A 1~GeV photon produces about 800 photoelectrons corresponding to a phototube
signal of about 0.5~V and a rise-time of 10~ns.  No further amplification is required.  
The signal will be digitized with an 8-bit 250~MHz FADC.

\subsubsection*{R\&D Issues, Simulations, Monitoring and Other Considerations}

The performance of the LGD has been described in two NIM publications for E852 
experiment\footnote{Nucl. Instr. \& Meth. \textbf{A332} 419 (1993); 
 Nucl. Instr. \& Meth. \textbf{A387} 377 (1997)} 
and a a submitted NIM article for Radphi. An earlier version of the CW base is described
in another NIM
publication\footnote{Nucl. Instr. \& Meth. \textbf{A414} 466 (1998)}.  GlueX R\&D has concentrated on
construction of 100 prototype improved CW bases, evaluation of  lead glass and FE-84-3
phototubes used in E852 and Radphi to
determining suitability for use in GlueX and various curing techniques to repair radiation damage of
lead glass.  Simulations of detector response is based on extensive
experience with E852 and Radphi data analysis.  Raphi experience is particularly important
as it involved operating an electromagnetic calorimeter in an bremsstrahlung
photon beam.
The monitoring system consists of a plastic scintillator sheet covering the 
up stream end of the glass stack
and illuminated by fibers connected to a pulsed laser.  



\subsubsection*{Collaboration Responsibilities}

The LGD is the responsibility of the groups from Indiana University and the Institute for
High Energy Physics (IHEP) in Protvino, Russia.  This manpower is adequate to complete
remaining R\&D in six months and to complete the detector construction (including CW bases) in
two years from availability of funds.


\subsubsection*{Purpose, Description}

The electronics will amplify, discriminate, and digitize raw detector signals
storing them for later readout at level 1 trigger rates of 200 kHz
without incurring deadtime.  The detector includes
approximately 18,600 ADC channels and 6,300 TDC channels.

A pipelined approach is required due to the high trigger rate.  The
digitized information must be stored for several $ \mu s $ while the
level 1 trigger is formed.  Multiple events must be buffered within
the digitizer modules and read or transmitted while the front ends continue to
acquire new events.  The raw data rate from the detector is about 1
Gbyte/second. A sophisticated timing system is required to synchronize the
pipelines in the front-end modules.  The plan is to clock all digitizers in synchronization
with the accelerator timing.

Energy sums from the calorimeters as well as track counts from the barrel and TOF systems
compromise the level 1 trigger.



\subsubsection*{Current Status, R\&D issues}


Since no currently available commercial solutions exist the 
preamps, discriminators, digitizer modules, and timing system
will be designed by GlueX.

JLab has designed a multi-channel TDC module based on the ACAM TDC-F1
chip, and 100 units are currently in use in existing JLab experiments.  A high density
FADC with auxiliary user-definable processing is currently being developed at
Jefferson Lab.

A single-channel prototype 8-bit 250~MHz FADC, suitable for the barrel
and lead glass calorimeters, has been constructed at Indiana
University.  A multi-channel version which includes the energy sum is
currently being designed.

 A prototype constant-fraction discriminator has been built at the University of Alberta.

Preliminary work has been done on the preamps and
the timing system, and on a slower FADC for the tracking chambers.




\subsubsection*{Review}

The GlueX electronics system was reviewed in July of 2003.  The
reviewers concluded the basic design is sound and appropriate assuming
adequate human and fiscal resources.

The reviewers recommended that the work on the TDC and FADC should
continue.  A multi-channel FADC needs to be prototyped and evaluated.
As part of the level-1 trigger, the digitized calorimeter signals are
summed in a pipelined adder tree; the prototype FADC needs to
demonstrate this capability.

Analog front-end requirements need to be settled.  Prototype work
needs to begin, especially on the tracking chamber electronics to be
located inside the magnet.  The pipelined level-1 trigger, timing,
synchronization, and calibration systems all need further development.



\subsubsection*{Collaboration Responsibilities}

Indiana University and JLab have the major responsibilities for the
FADCs and TDCs, respectively. Still, 
the reviewers concluded that current manpower resources are
inadequate.  The University of Alberta has recently joined the
collaboration, but additional institutions possessing electronics
expertise are needed.  Discussions are under way with the Indiana University
Cyclotron Facility.

The reviewers also noted that a rudimentary management plan exists, but
that it needs further development.  
They estimated that the GlueX electronics effort could
require 6 years to complete after CD-3; the collaboration hopes to
reduce this time.  One possibility is a commercial partner/collaborator.

%\documentclass[10pt]{article}
\usepackage{graphicx}
\usepackage{amssymb}
\usepackage{epstopdf}
\usepackage[]{epsfig}
\DeclareGraphicsRule{.tif}{png}{.png}{`convert #1 `basename #1 .tif`.png}

\textwidth = 7.0 in
\textheight = 9.5 in
\oddsidemargin = -0.25 in
\evensidemargin = 0.0 in
\topmargin = 0.0 in
\headheight = 0.0 in
\headsep = 0.0 in
\parskip = 0.05in
\parindent = 0.0in

\def \fdc    {{\textsc{fdc}}}

\begin{document}

\documentclass[10pt]{article}
\usepackage{graphicx}
\usepackage{amssymb}
\usepackage{epstopdf}
\usepackage[]{epsfig}
\DeclareGraphicsRule{.tif}{png}{.png}{`convert #1 `basename #1 .tif`.png}

\textwidth = 7.0 in
\textheight = 9.5 in
\oddsidemargin = -0.25 in
\evensidemargin = 0.0 in
\topmargin = 0.0 in
\headheight = 0.0 in
\headsep = 0.0 in
\parskip = 0.05in
\parindent = 0.0in

\def \fdc    {{\textsc{fdc}}}

\begin{document}




\section*{CNU trigger development}


\subsubsection*{Overview}

	CNU has committed to developing the VXS-based Energy Sum Slot A (VESSA) board, 
which will reside in Slot A of a VXS backplane.  This board receives Energy Sum 
information over the VXS serial backplane lines from each of the FADC boards in the crate.  
It combines (essentially adds) the information from all boards in its crate to create 
a Crate Energy Sum, which will be sent to the Global Level 1 Trigger Processor.

	To perform its task, the VESSA will use multiple FPGAs, arranged in a three 
stage pipeline.  With a total of 18 payload slots (nine on each side) available in a 
VXS backplane, the first stage uses a total of six FPGAs to handle the 
serial input data, with each FPGA handling the data from three slots (FADCs).  Each 
of these inbound FPGAs needs to have  multiple Multi-Gigabit Transceivers (MGTs) to 
receive the data from the VXS backplane.  Each of the inbound FPGAs will aggregate the data from their 
boards, then pass the data to one of two second stage FPGAs (three to one, 
three to the other).  Each of these second stage FPGAs will then aggregate the 
data from each side.  This data will be passed to the final summing stage, which 
will also serve as the outbound link manager.


\subsubsection*{Stages of  R\&D}

	The development of a completed VESSA board is anticipated to take place in four stages.

	The first stage is the development is to design and layout VESSA-1, used 
for prototyping communications over the backplane.  It contains a single FPGA, and 
sends data on several (but by no means all) of the VXS backplane links to several of 
the slots in the crate.  A companion loopback board will also be designed, which can 
be moved to different slots, to reflect the signal back to the VESSA-1 board
The second prototype, VESSA-2, is used to aggregate the data from multiple FADC boards 
in one crate.  It will not handle the full 18 slots, and it will not have an outbound 
serial link, but it will handle at least one side of nine slots, and have a two stage 
pipeline (the third stage is not necessary) with a trigger output signal.  This is a 
fully functional board which will be used in a prototype detector system.

	The third stage of the prototyping is a prototype of the outbound link system, 
currently anticipated to be a fiber optic version of the Slink-64.

	The final stage is to integrate the Slink-64 with an enhanced VESSA-2 to 
create the VESSA-3, which handles up to 18 FADC boards, and has the outbound link 
to the Global Trigger Processor.

\subsubsection*{Estimated cost}

	Thus the development of the completed VESSA board requires at least five boards 
of differing complexity to be developed.  It should be noted that each of these boards 
needs to be controlled impedance.  The anticipated cost for developing these five boards, 
along with associated components and assembly cost is roughly \$25,000.  With a 50\% 
contingency, we arrive at a final cost of \$37,500.	


\end{document} 

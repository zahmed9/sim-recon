%\documentclass[10pt]{article}
\usepackage{graphicx}
\usepackage{amssymb}
\usepackage{epstopdf}
\usepackage[]{epsfig}
\DeclareGraphicsRule{.tif}{png}{.png}{`convert #1 `basename #1 .tif`.png}

\textwidth = 7.0 in
\textheight = 9.5 in
\oddsidemargin = -0.25 in
\evensidemargin = 0.0 in
\topmargin = 0.0 in
\headheight = 0.0 in
\headsep = 0.0 in
\parskip = 0.05in
\parindent = 0.0in

\def \fdc    {{\textsc{fdc}}}

\begin{document}

\documentclass[10pt]{article}
\usepackage{graphicx}
\usepackage{amssymb}
\usepackage{epstopdf}
\usepackage[]{epsfig}
\DeclareGraphicsRule{.tif}{png}{.png}{`convert #1 `basename #1 .tif`.png}

\textwidth = 7.0 in
\textheight = 9.5 in
\oddsidemargin = -0.25 in
\evensidemargin = 0.0 in
\topmargin = 0.0 in
\headheight = 0.0 in
\headsep = 0.0 in
\parskip = 0.05in
\parindent = 0.0in

\def \fdc    {{\textsc{fdc}}}

\begin{document}




\section*{Offline Software at JLab} 


\subsubsection*{Overview}

The offline software effort for GlueX was reviewed extensively during
the Software Workshop on October 21-22, 2005. 
At that time  the present
status was reviewed and near and longer term goals were discussed. 
The workshop was summarized in GlueX-doc-537.  
Milestones 
for this system were summarized at the collaboration meeting. 


\subsubsection*{FTE for software framework}

A framework exists that has been written specifically for GlueX
reconstruction (DANA). However, there are a number of
features/improvements that still need to be developed and incorporated
into the existing framework. These include: Logging system, Exception
handling, Object introspection, Improved thread management, Interface to
ET system for monitoring, Calibration/Geometry database access, and
Interface to PWA code. Development of a program that implements a Hall-A
style histogram definition file is also needed to provide access to the
analysis software to a wider group of GlueX collaborators as well as
improve the efficiency in which analyses are performed. Integration of
the framework into the L3 trigger is also needed along with extensive
testing.


\subsubsection*{FTE for charged particle tracking}

At this point, code has been written that does an initial track finding
pass on parametrically smeared Monte Carlo data using a homogeneous
magnetic field with some noise hits added. The next step will be to
develop a Kalman filter than can propagate a track through a
inhomogeneous magnetic field, using covariance matrices passed from the
underlying detector systems to describe the hits. This will require
identifying and integrating a linear algebra package into the GlueX
software framework, implementing code to read in and interpolate the
magnetic field and coordinating with the authors of the  reconstruction
code for the  individual detector systems.  Proper  application of the
Kalman filter will also require integration of geometry and materials
information in order to calculate process noise due to scattering from
material in the particle path. The tracking code will also need to
accommodate both off beam-line vertex reconstruction and re-fitting
using information
derived from particle ID kinematic fitting code.


Thee following tasks have been identified as goals for the offline
software effort:

\begin{enumerate}
\item ``Phase 1" Tracking and particle ID. Output into format that can
by read by CMU PWA code.
   \begin{itemize}
    \item Pattern recognition using hit-based information completed and optimized for Monte Carlo data.
    \item Full tracking through an inhomogeneous magnetic field. The Kalman
   filter should be implemented at least in a rudimentary way.
    \item Particle identification factory designed with well defined outputs.
   Not necessarily final algorithm, but at least capable of identifying probable Kaons.
   \end{itemize}                                                                                            
\item ``Phase 2" Tracking and particle ID.
  \begin{itemize}
   \item Off beamline vertex reconstruction.
   \item Improved particle ID through kinematic fitting.
   \end{itemize}
\end{enumerate}


\subsubsection*{Goals and Milestones}


\subsubsection*{Personnel}

Primary responsibility for developing the software framework belongs to David Lawrence.
Although preliminary track finding and fitting has been studied using the
Monte Carlo ``truth tags'', a substantial effort is still required for full reconstruction
and requires an additional person.


\end{document}

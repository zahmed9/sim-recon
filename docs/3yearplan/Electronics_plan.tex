%\documentclass[10pt]{article}
\usepackage{graphicx}
\usepackage{amssymb}
\usepackage{epstopdf}
\usepackage[]{epsfig}
\DeclareGraphicsRule{.tif}{png}{.png}{`convert #1 `basename #1 .tif`.png}

\textwidth = 7.0 in
\textheight = 9.5 in
\oddsidemargin = -0.25 in
\evensidemargin = 0.0 in
\topmargin = 0.0 in
\headheight = 0.0 in
\headsep = 0.0 in
\parskip = 0.05in
\parindent = 0.0in

\def \fdc    {{\textsc{fdc}}}

\begin{document}

\documentclass[10pt]{article}
\usepackage{graphicx}
\usepackage{amssymb}
\usepackage{epstopdf}
\usepackage[]{epsfig}
\DeclareGraphicsRule{.tif}{png}{.png}{`convert #1 `basename #1 .tif`.png}

\textwidth = 7.0 in
\textheight = 9.5 in
\oddsidemargin = -0.25 in
\evensidemargin = 0.0 in
\topmargin = 0.0 in
\headheight = 0.0 in
\headsep = 0.0 in
\parskip = 0.05in
\parindent = 0.0in

\def \fdc    {{\textsc{fdc}}}

\begin{document}



\section*{GlueX Electronics for DAQ}


\subsubsection*{Overview}

The GlueX DAQ electronics plan encompasses both analog and digital requirements and will be 
responsible for discriminating, and digitizing raw detector signals
storing them for later readout at level 1 trigger rates of 200 kHz
without incurring deadtime.  The GlueX detector includes approximately 18,600 ADC 
channels and 4,300 TDC channels. Additional DAQ related electronics must be developed for
collecting energy sums and track counts in order to process the L1 trigger decision, and
then to distribute this trigger information to upwards of 70-80 front-end crates. Other
electronics needs including preamps and other detector specific needs are not covered
in this document.
 
A pipelined approach will be required to support the high trigger rate.  The
digitized charge and time information must be stored for several $ \mu s $ while the
level 1 trigger is formed and then distributed.  Multiple events must be buffered within
the digitizer modules and read or transmitted while the front ends continue to
acquire new events.  The aggregate raw data rate from the GlueX detector is expected to
reach  about 1 Gbyte/second. A sophisticated timing system is required to synchronize the
pipelines in the front-end modules.  The plan is to clock all digitizers in synchronization
with the accelerator timing.

In parallel, formed energy sums from the calorimeter crates as well as track counts from the 
barrel and TOF systems must be collected and clocked back to the Master L1 trigger crate/module 
where the data and logic that compromise the level 1 trigger can be processed.

\begin{figure}[p]
\begin{center}
\includegraphics[height=15cm]{Electronics_plan.eps}
\caption{Schematic of the electronic boards that are required to support
the plans for GlueX pipeline electronics and DAQ. 
Indicated are the crates needed, responsible groups,
expected use in the 6 GeV program, and estimate of time frame for design and
prototyping.
\label{fig:plan}}
\end{center}
\end{figure}



\subsubsection*{Stage One R\&D}

There are a number of parallel board development projects which must progress over the next 3
years both to support existing DAQ requirements and to provide R\&D for future designs.
In total to date 8 board designs have been identified, in rough order of development:
\begin{itemize}
\item 250 MHz 16 Channel Flash ADC (Ver 1 VME/VXS)  FY06 FY07
\item VME Trigger Interface (Ver 3 VME only)        FY06
\item F1TDC 64/32 Channel (Ver 2 VME/VXS)           FY06 FY07
\item Backplane Distribution Card (Ver 2 VME only)  FY06
\item Crate Energy Sum/Trigger Process (Ver 1 VXS)  FY06 FY07 FY08
\item VXS Trigger/Clock Interface (Ver 1 VXS)       FY07 FY08
\item Trigger Distribution Board (Ver 1 VME/VXS)    FY08 FY09
\item Master L1 Trigger Processor (Ver 1 VME/VXS)   FY08 FY09
\end{itemize}

The DAQ group is involved jointly with the JLAB electronics group in the 
development of all of the above designs. These boards can be roughly catagorized into
three different subsystems: Digitizing, L1 Trigger Processing, and Trigger/Clock Distribution.
However, they all must provide cross support for communication between subsystems.
All experimental halls in the 12 GeV upgrade plan are expected to use modules in both
the Digitizing and/or Trigger/Clock Distribution Systems. Therefore these boards are going
to be developed without requiring the VXS backplane support.

With the design for the 250 MHz FADC underway, simultanious development must be done on
a new VME Trigger Interface (TI Board Ver 3). The JLAB trigger interface card is a 
critical hardware component used to support many aspects of the CODA data 
acquisition system, specifically in the front-end, but it also provides data 
to the backend for event-building. It is used in both single and multi-crate systems 
and provides both trigger information and general I/O to the ROC software component.

As the CODA DAQ system evolves to support higher rate experiments, pipelined 
front-ends, and soft real-time systems via embedded Linux, the capabilities 
of this module must evolve as well. Below are outlined the general features 
of the version 3 trigger interface that are required.

\begin{itemize}
\item Software and hardware compatibility with TI version 2. Must work with the 
existing Trigger supervisor preferably with minimal modification of existing 
software.
\item Include remote VME Reset capability currently provided by separate 
microcontroller card.
\item Support trigger buffering locally. Record high resolution time stamp with 
each accepted trigger. Creation of a trigger event bank in RAM as events are 
accepted.
\item VME slave must support block readout of trigger event bank information. 
Can be used with F1 and FADC in chained block reads.
\item Onboard scalers available, trigger count and connected to user inputs.
\item VME Master capability. For example upon receiving a trigger the card would 
have the ability to go on the bus and read module status or latched scaler 
and buffer this info in the trigger event bank.
\item Clock/Trigger source for F1TDC and/or FADC - linked to time stamp. Lines to 
P2 for input to the backplane distribution card. 
\item User ability to program local FPGA. Appropriate code should be loadable 
over VME and then be able to downloadable into the FPGA. This code should 
then be selectable to be executed over the standard trigger processing code. 
This effectively becomes the "primary" user readout list for the crate.
\end{itemize}

The addition of updated versions of the Backplane Distribution Card and F1TDC Board 
will complete an initial phase of electronics development in that a consistant DAQ
system that supports pipelining, and event blocking will be available. It will also
be supported by the existing JLAB Trigger Distribution System (Trigger Supervisor) and
be useable by existing 6 GeV experiments. Most of the GlueX detector development DAQ
will also be supported by this system.

Ideally this stage of development should be completed by the end of FY07.


\subsubsection*{Stage Two R\&D}

Traditional DAQ systems typically split detector signals and send one set for
discrimination and L1 trigger processing, and the other set for digitization. With
the availability of pipeline electronics and high speed flash ADCs one now has
the ability for the digitizers to do both, requiring only a single input
from the detector and no delay cabling. 

We plan to use a commerically supported extension to VME called VXS (VME Switched Serial)
which provides a full 8 lane crossbar fabric interconnecting 18 VME payload slots in a crate 
with two specialized switched slots. These data lanes currently support 3 Gbit/s transfers
and are expected to increase to 10 GBit/s. Using this technology we intend to design two
additional switch slot crate level boards. One, the Crate Sum/Trigger Processor, will collect
partial sums from all FADCs in the crate and perform addtional sums and/or trigger processing.
This board would also be responsible for sending trigger data to a final Master Trigger board
elsewhere in the GlueX experimental Hall. The second board would be a forth revision of the
Trigger Interface card only this board would reside in a switch slot so as not to take up a 
payload slot. In addition the trigger and clock signals would be distributed to all payload
boards via the VXS backplane. No backplane distribution card should be necessary in this
configuration.

The two final subsystems that will be developed (in parallel with the VXS crate cards) are the
Master L1 Trigger Subsystem and the Trigger/Clock Distribution System (Trigger Supervisor
replacement). The design of these boards cannot be fully specified at this time. Much of this
will depend on the R\&D being done with VXS. The form factor necessary to collect signals from
a large number of crates (around 20-30) and process that data within a few microsecs is not
known. The Trigger/Clock distribution board however is expected to fit in a 6 or 9U VME form factor
as it should be available for use by all experimental halls regardless if they intend to 
use the L1 trigger processor or not. There is no reason to expect to have to design two different
Trigger Distribution Boards, so consideration should be given to the design to allow a 
compatable interface between it and the output of Master L1 Trigger board.


\subsubsection*{Goals Milestones and Personnel}

The high speed FADC is expected to be in prototype by the summer or early fall of 2006,
The new VME trigger interface (Version 3) capable of supporting both the old
trigger supervisor as well as the new pipelined front-end should be prototyped in the
same timeframe. This board along with the Backplane Distribution card must be designed
to fully support and work cooperatvely with the FADC. Once these systems have been 
sufficiently prototyped, version two for the F1TDC can be specified and designed. 

By early 2007 there should be enough hardware and software development to
support small detector system prototyping, and beam tests that would
encompass the core functionality of the GlueX DAQ system. By the end of FY07 we should
have available a complete set of hardware to support fully pipelined high-rate
systems at least at the single crate level.

VXS R\&D should be sufficiently advanced by the summer of 2007 to begin protoyping the
Crate Sum and VSX Trigger interface cards. Prototypes could be available in 2008.

Personnel working on these projects include the JLAB electronics group as well as Ed Jastrzembski
of the DAQ Group. There are currently 3 design engineers working on these projects all part time.
A 4th design engineer will be needed to sufficently cover the overlaping requirements of these
designs. In addition as prototyping moves to production, more help will be needed in
the form of an electrical tech that will be able to test, debug and repair. Support for all
these designs will be ongoing as they are produced.


\end{document} 

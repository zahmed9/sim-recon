
\documentclass[12pt]{article}
\usepackage{epsfig}
\usepackage{anysize}
\usepackage{caption2}
\usepackage{fancyvrb}
\usepackage{listings}
\usepackage{color}

\lstset{basicstyle=\footnotesize,fancyvrb=true}
\definecolor{codebkgd}{rgb}{0.95,0.95,1.0}
\definecolor{cmdbkgd}{rgb}{0.95,1.0,0.95}

% Set reasonable margins
\marginsize{2cm}{2cm}{3cm}{2cm}
\setcaptionwidth{5.0in}

% These commands try to force figures to co-exist with text
% better than the default which tends to throw figures onto
% their own pages.
\renewcommand{\topfraction}{0.85}
\renewcommand{\textfraction}{0.1}
\renewcommand{\floatpagefraction}{0.75}

\author{D. Lawrence \\ Jefferson Lab}
\title{Getting Started with DANA}
\begin{document}

\maketitle

%\vspace{-6cm}
%\rightline{Hall-D-NOTE ??}
%\vspace{6cm}

%----------------------- Abstract -------------------------
\abstract{The Hall-D {\bf D}ata {\bf ANA}lysis framework (DANA) provides the
mechanism by which various pieces of the reconstruction software
are brought together to fully reconstruct the data. This
document is intended for user's new to the DANA system. This
includes both end user's who simply want to get at the data for
their own analysis, and person's writing reconstruction code
for a particular subsystem. It is purposely kept short to minimize
you TTR (Time To Results).}

\newpage
\tableofcontents
\newpage

%----------------------- Introduction -------------------------
\section{Introduction}
The DANA system is centered around the idea of data factories. In the
DANA system, a factory is an object which takes one or more inputs
and produces a single output. The inputs and outputs are lists of objects.
So, for example, if I wanted to produce a list of showers in the forward
lead glass calorimeter (FCAL), I would first need a list of hits in the
FCAL. Similarly, in order to produce a list of tracks, I would first
need a list of hits in the CDC, FDC, etc. In general, factories get their
inputs from other factories. End point users (persons doing an analysis
and not trying to produce a list for anyone else) get at the data in the
same way, by accessing the factory.

The factory objects themselves provide ``data on demand''
\footnote{This concept is also called ``just in time''.}. What this
means is that when an event is read in, nothing is analyzed until the
data is actually requested from the factory. If the program being run
is only interested in Barrel Calorimeter (BCAL) data, then there is
no need to do full scale tracking or FCAL reconstruction or anything
else other than BCAL related reconstruction on the event.
Since factories get their input from other factories, a heirarchy
of factories is naturally formed with each knowing what factories it
needs to access for its inputs.



%----------------------- Getting and Compiling the Source Code -------------------------
\section{Getting and Compiling the Source Code}
The Hall-D reconstruction code is kept in a CVS repository on the 
Hall-D group disk (/group/halld) at Jefferson Lab. To access it,
you need an account on the JLab CUE which belongs to the halld unix
group\footnote{contact the JLab Computer Center if you need don't
already have such an account}. Follow these steps to checkout and compile
the code:

\begin{itemize}
	\item{\bf Create working directory} All of the source code and binaries
	will reside in this directory. It can be named anything and placed 
	anywhere. In my account, I use $/home/davidl/HallD$.
	
	\item{\bf Set HALLD\_HOME environment variable} The HALLD\_HOME
	variable should
	be set to the working directory you just made. The makefile system
	\footnote{See GlueX-doc-473 on the BMS system}
	uses this to find the source code and place the resulting binaries.
	So, for my account this is set to $/home/davidl/HallD$.
	
		\begin{lstlisting}[frame=shadowbox,backgroundcolor=\color{cmdbkgd}]
setenv HALLD_HOME /home/davidl/HallD
		\end{lstlisting}

	(I use tcsh in these instructions I'll leave it to $bash$ users to translate
	where appropriate.)

	\item{\bf Checkout the source} Go into
	your working directory and checkout the code by doing the following:
	
		\begin{lstlisting}[frame=shadowbox,backgroundcolor=\color{cmdbkgd}]
cd $HALLD_HOME
setenv CVS_RSH `which ssh`
setenv CVSROOT cvs.jlab.org:/group/halld/Repositories/cvsroot
cvs co src include
		\end{lstlisting}

	Note that this assumes the account you're issuing the command from
	has the same username as your JLab CUE account. If not, prefix
	$CVSROOT$ with your JLab account name followed by an `@'
	\footnote{e.g. joe@cvs.jlab.org:/group/halld/Repositories/cvsroot}.

	\item{\bf Move the $include$ directory} Hopefully this step will be
	removed in the near future. But for now, you'll need to move the
	include directory you just checked out into the src directory
	(that you also just checked out).

		\begin{lstlisting}[frame=shadowbox,backgroundcolor=\color{cmdbkgd}]
mv include src
		\end{lstlisting}
	
	\item{\bf Compile the libraries} Go into the $src/libraries$ directory
	and run gmake:

		\begin{lstlisting}[frame=shadowbox,backgroundcolor=\color{cmdbkgd}]
cd src/libraries
gmake
		\end{lstlisting}

	This should build all of the libraries and place them in
	{\it \$HALLD\_HOME/lib/\$OSNAME} where $\$OSNAME$ is the uname of
	the system you're working on (e.g $Linux$).

	\item{\bf Compile the executables} Go into the {\it src/programs/Analysis}
	and run $gmake$ in the $hd\_dump$, $hd\_ana$, and $hdview$ directories:

		\begin{lstlisting}[frame=shadowbox,backgroundcolor=\color{cmdbkgd}]
cd $HALLD_HOME/src/programs/Analysis
gmake -C hd_dump
gmake -C hd_ana
gmake -C hdview
		\end{lstlisting}
		
	The executables will be placed in the {\it \$HALLD\_HOME/bin/\$OSNAME}
	directory. These programs do the following:
	\begin{itemize}
		\item{\it hd\_dump} Dump ASCII output to screen from all factories
		\item{\it hd\_ana} Generates a couple of example $ROOT$ histograms
		\item{\it hdview} Graphical drawing of data
	\end{itemize}
	
\end{itemize}

%----------------------- Running the software -------------------------
\section{Running the software}
All of these programs use the same command line parser routine
that is part of DANA. This just means, is that any command
line argument that does not begin with a ``-'' is assumed to be 
a data file\footnote{At this point in time, the only format we have
for data files is HDDM}. Each of the programs will loop over
all events in all of the data files specified on the command line.
See the sections below for examples.

%----------------------- Generating a Data File -------------------------
\subsection{Generating a Data File}
If you already have an HDGeant produced data file then you can 
probably skip this section. Note though that HDDM can be very
picky about version numbers in files so if you have an old file,
the executables may simply exit with an error.

Here, I'll assume you have a working $hdgeant$ executable.
For the sake of simplicity, I'll show how to use the 
single particle generator built into $hdgeant$. The only thing
you really need is a file called $control.in$ in the directory
you run hdgeant from. Here's an example one that throws 1.5GeV
$\mu^{+}$s from the center of the target (z=65cm) at randomly
chosen angles. See the $GEANT$ documentation for descriptions
of the uncommented ``cards''.

\begin{lstlisting}[caption=control.in,frame=shadowbox,backgroundcolor=\color{codebkgd}]
TIME 1. 1. 1000000
TRIG 1000000
RUNG 9999

C The following card enables single-track generation (for testing)
C Note: if you REALLY want the specified theta/phi for every event,
C you must add 100 to the particle type. Otherwise, theta and phi
C will be randomly chosen.
C    particle  energy  theta  phi  vertex(x y z)
KINE    5      1.5     10.  265.    0. 0. 65.

RNDM 172

CUTS 1e-4 1e-4

SWIT 0 0 0 0 0 0 0 0 0 0

C The following card enables the GelHad package (from BaBar)
C   on/off  ecut   scale   mode   thresh
GELH  1     0.2     1.0     4     0.160

HADR 1
CKOV 1
LABS 1
LOSS 1
MULS 1
MUNU 1
PAIR 1
PFIS 1
PHOT 1
RAYL 1
STRA 1
SYNC 1

C The NOSECONDARIES card is specific to HDGeant. Setting it to a non-zero
C value will skip calls to GSKING()
NOSECONDARIES 1

C The BFIELD card is specific to HDGeant. Setting it to some value
C (in Tesla) will cause the tracking to use a completely homogeneous
C magnetic field with this value in the Z (beam) direction. Otherwise,
C the TOSCA generated field map(dsolenoid.table) will be used.
BFIELD -2.0

END
\end{lstlisting}

Run $hdgeant$ without any arguments. It will produce several files,
but the only one you're interested in is called $hdgeant.hddm$.

%----------------------- hd_dump -------------------------
\subsection{{\it hd\_dump}}
The $hd\_dump$ program is a very basic one that can be used to dump
single events in ASCII format to the terminal. Run it with no arguments
to see a usage statement. Basically, to see the output of any factory,
just use the $-D$ option. For example, the following will dump the contents
of both the $DBCALHit$ and $DFCALHit$ factories:

\begin{lstlisting}[frame=shadowbox,backgroundcolor=\color{cmdbkgd}]
hd_dump -DDBCALHit -DDFCALHit hdgeant.hddm
\end{lstlisting}

This program is useful just to see that you're able to read the
data. It relies on the factory objects themselves to format the
output since factories are not required to implement a
``toString()'' method, not all of them will produce useful output
from $hd\_dump$.

%----------------------- hd_ana -------------------------
\subsection{{\it hd\_ana}}
The $hd\_ana$ program is intended to be the definitive example
program for end user analysis. It opens a ROOT file, creates a
few histograms, and then fills them using data obtained through
$DANA$. If you're writing a new analysis program, you probably just
want to copy the source from this directory and edit from there.
It is worth compiling and running to test that things are working,
but the output isn't terribly useful beyond that. Run it like this:

\begin{lstlisting}[frame=shadowbox,backgroundcolor=\color{cmdbkgd}]
hd_ana hdgeant.hddm
\end{lstlisting}

%----------------------- hdview -------------------------
\subsection{{\it hdview}}
The $hdview$ program provides a simple graphical representation
of events. It was developed to help visualize things for tracking
and so is of limited use for other systems. However, it may be
of some diagnostic use. Run it just like all of the other programs:

\begin{lstlisting}[frame=shadowbox,backgroundcolor=\color{cmdbkgd}]
hdview hdgeant.hddm
\end{lstlisting}


%----------------------- Adding your own code -------------------------
\section{Adding your own code}
At this point, you should have a working copy of the Hall-D source 
code and an $hdgeant$ produced data file ($hdgeant.hddm$) to work with.
Modifying the source can be broken up into two groups:
\begin{itemize}
	\item{\bf End User} Projects in this group want to perform
	data analysis for a special study. These projects do not result
	in output data which can be used either by factories or other
	end users. These are executable programs. See section \ref{enduser}.
	
	\item{\bf Factories} Projects in this group perform some piece
	of reconstruction or analysis that is made available to other
	factories or end user's. These are kept in libraries, usually with
	other related factories. See section \ref{factory}.
\end{itemize}

%----------------------- I just want to make a histogram! -------------------------
\subsection{``I just want to make a histogram!''}
\label{enduser}
Oh you do, do you? Good, you've come to the right place. The best
tutorial for this is actually to look at the example program $hd\_ana$. The
source for this program is kept in
{\it \$HALLD\_HOME/src/programs/Analysis/hd\_ana}. For convienience, the
source has been included in this note in appendix \ref{hd_ana_src}.
The basic idea is
this: You need to create a class which inherits from the $DeventProcessor$
class. You can then overide any of the methods corresponding to the 5 phases
of the data processing:

\begin{itemize}
	\item{\bf init(void)} Called once at initialization
	\item{\bf brun(int runnumber)} Called anytime the run number changes.
	 Guaranteed to be called once before the first call to $evnt()$.
	\item{\bf evnt(int eventnumber)} Called every event.
	\item{\bf erun(void)} Called when the run number changes (iff $brun()$
	 was already called) or after all events have been processed.
	\item{\bf fini(void)} Called once after all events have been processed.
\end{itemize}

You create a single instance of your derived class and pass a pointer
to it to the framework. Control is then
handed over to the framework which will call the above methods of your
class at the appropriate times.

Inside your $evnt()$ method, you will request data objects from the 
framework. The framework will look through it's list of factories
until it finds the one which produces the requested data type and returns
a list of those objects to you.

%----------------------- Writing a Factory -------------------------
\subsection{Writing a Factory}
\label{factory}
If you are writing reconstruction code that produces output for others
to use, then you should write a factory. A factory consists, at a
minimum, of 3 files which define: 1. the produced object header.
2. the factory object header, and  3. the factory object methods.
These can be produced using the $mkfactory$ script which resides in
the $scripts$ directory in CVS
\footnote{cvs -d cvs.jlab.org:/group/halld/Repositories/cvsroot co scripts/mkfactory}.
Invoke the script with the name of the data object you'd like to
generate with the factory. For example:

\begin{lstlisting}[frame=shadowbox,backgroundcolor=\color{cmdbkgd}]
mkfactory DParticle
\end{lstlisting}

would produce 3 files named $DParticle.h$, $DFactory\_DParticle.h$,
and $DFactory\_DParticle.cc$.\\

The header files should be copied to the $\$HALLD\_HOME/src/libraries/include$
directory. The source should be placed in the appropriate subsystem
directory. For example, a factory dealing with BCAL should be placed
in the $\$HALLD\_HOME/src/libraries/BCAL$ directory. In that same
directory, there is a file named something like $BCAL\_init.cc$
which looks like this:

\lstinputlisting[caption=BCAL\_init.cc,label=BCAL_init.cc,frame=shadowbox
	,backgroundcolor=\color{codebkgd}]{BCAL_init.cc}

$BCAL\_init()$ (as well as all of the other subsystems' $*\_init$
routines) gets called when the $DEventLoop$ object is instantiated in
the $main()$ routine of all programs using DANA. This is how your
factory gets added to the list of factories kept by the $DEventLoop$.\\

The place where you actually want to place your code for the
factories is in the $init()$, $brun()$, $evnt()$, $erun$, and $fini()$
methods\footnote{see section \ref{enduser}} of your new DFactory class.
The most common usage, however, will be to read calibration constants in
$brun()$ and then do the actual reconstruction in $evnt()$.\\

The factory will produce data in the form of objects. More specifically,
the factory will fill an STL $vector$ of pointers to objects
which the factory creates. The DANA framework will automatically delete
these objects when the next event is read in. So, the first thing
you need to do is finish the definition of the class by adding the
data fields. Listing \ref{DMCReconstructed.h} shows an example of a class
$DMCReconstructed$ produced by a factory. The $mkfactory$ script created
everything in this file except the lines after $HDCLASSDEF(DMCReconstructed)$.
The $HDCLASSDEF(DMCReconstructed)$
is a DANA defined macro that places a couple of methods in the
class which it needs. It must be placed in the public area of the class.\\

Notice that each data element in listing \ref{DMCReconstructed.h} has a
comment prefaced with ``$///<$''. This special comment delimiter
is used by the Doxygen documentation system to present the class
definition in the web-based documentation in the proper place.\\

\lstinputlisting[caption=DMCReconstructed.h,label=DMCReconstructed.h,frame=shadowbox
	,backgroundcolor=\color{codebkgd}]{DMCReconstructed.h}


The next step is to create objects in your factory's $evnt()$ method
and add them to the $\_data$ vector of your factory. The variable
$\_data$ is a vector whose type is based on the class of the objects
the factory creates. Listing \ref{DFactory_DMCReconstructed.cc} shows
the $evnt()$ method for the factory that creates the $DMCReconstructed$
objects. This factory accesses data from two other factories in
order to produce it's output. The key thing to notice that near the bottom
of the routine is a line $\_data.push_back(mcreconstructed)$. This
is how the objects are stored so that DANA can access them. DANA
will take care of casting the pointers as $const$ before delivering them
to others. DANA will take care of deleting the objects later, when
the next event is read in.\\

\lstinputlisting[caption=DFactory\_DMCReconstructed.cc,label=DFactory_DMCReconstructed.cc,frame=shadowbox
	,backgroundcolor=\color{codebkgd}]{DFactory_DMCReconstructed.cc}


The bottom half of listing \ref{DFactory_DMCReconstructed.cc} is
the factory's $toString()$ method. This is used by $hd\_dump$
to print the factory contents to the screen. A few routines are
provided to the factory by the $DFactory_base$ base class to
help make this easier. These are $printheader$, $printnewrow$,
$printcol$, and $printrow$. The example illustrates how these
are used.

%----------------------- Questions -------------------------
\section{Questions}
Please refer any questions to David Lawrence at $davidl@jlab.org$.

%--------------- Appendicies ----------------
\newpage
\appendix
\section{Source for $hd\_ana$ program}
\label{hd_ana_src}

%-------------- hd_ana.cc -------------
Listing \ref{hd_ana.cc} shows the {\it main()} routine for the
hd\_ana program. It only has three lines. It just creates a MyProcessor
object and a DEventLoop object and then it tells the DEventLoop object
to run over all events, calling the methods of the MyProcessor object
when needed. Notice that the command line arguments are given to 
the DEventLoop object when it is created. It will parse them for
,among other things, the file names to read events from.

\lstinputlisting[caption=hd\_ana.cc,label=hd_ana.cc,frame=shadowbox
	,backgroundcolor=\color{codebkgd}]{hd_ana.cc}

%-------------- MyProcessor.h -------------
\newpage
Listing \ref{MyProcessor.h} shows the header file which defines the
$MyProcessor$ class. It defines 5 routines corresponding to the 5
phases of event processing described in section \ref{enduser}. Here,
the $brun()$ and $erun()$ routines are defined right in the header
to do nothing and just return no error\footnote{$NOERROR$ is defined
in $derror.h$ which is included from $DEventProcessor.h$.}. Pointers
for the ROOT file and a few histograms are defined here as well.

\lstinputlisting[caption=MyProcessor.h,label=MyProcessor.h,frame=shadowbox
	,backgroundcolor=\color{codebkgd}]{MyProcessor.h}

%-------------- MyProcessor.cc -------------
\newpage
Listing \ref{MyProcessor.cc} is the file that contains the real meat
of the program. The $init()$, $evnt()$, and $fini()$ methods
are defined. I'll describe each of them a little here:

\begin{description}
\item{\bf derror\_t MyProcessor::init(void)}
This routine is where the ROOT file is opened and the histograms are
defined. The first line: {\it eventLoop$->$PrintFactories();} just
dumps a list of all of the registered factories to the screen. It isn't
required. The rest of the lines are just ROOT. See the documentation
for ROOT for more details.

\item{\bf derror\_t MyProcessor::evnt(int eventnumber)}
This routine gets called every event and is where the histograms
are filled. In this example, two types of data are requested from
the $DEventLoop$ object, $eventLoop$\footnote{The $eventLoop$ variable
is part of the $DEventProcessor$ class and so is inherited by
the $MyProcessor$ class. It is set by DANA before calling any of
the $MyProcessor$ methods.}. First, the variables $cdchits$ and
$fcalhits$ are declared\footnote{Note that since we are asking for
the DCDCHit and DFCALHit objects, their respective header files are
included at the top of the file.}. Next, the eventLoop is asked to
get these data types from whatever factory produces them.

In the next section, there are two $for()$ loops which loop over the elements returned by
the $eventLoop->Get()$ calls, filling the histograms.

The last line, $eventLoop->PrintRate();$, updates the event processing
rate on the screen. It only prints to the screen about once per second
and so is safe to call every event. It is not required though.

\item{\bf derror\_t MyProcessor::fini(void)}
This routine just flushes the histograms to the ROOT file and then closes it.
\end{description}

\lstinputlisting[caption=MyProcessor.cc,label=MyProcessor.cc,frame=shadowbox
	,backgroundcolor=\color{codebkgd}]{MyProcessor.cc}

\end{document}



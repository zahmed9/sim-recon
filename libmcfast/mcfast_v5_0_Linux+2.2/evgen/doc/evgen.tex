\documentstyle[12pt,epsf]{article}
\textwidth 465pt
\textheight 683.4pt
\oddsidemargin 0pt
\evensidemargin 0pt
\topmargin -2cm
\pagestyle{plain}
\def\dec{\rightarrow}
\def\qsq{$q^2$}

\begin{document}

\title{The Event Generator}

\author{P. Avery, L. Garren}

\maketitle
 

\section{Introduction}
 
Evgen provides as consistent a wrapper as possible around the Isajet, 
Pythia, and Herwig Monte Carlo event generators.  Each example program contains
the option of decaying the particles via QQ.  The example code  
can be found under \$EVGEN\_DIR/example.
All examples fill the HEPEVT common block and write events to disk
using the stdhep xdr (mcfio) I/O routines.  The examples are designed
to generate b events at collider energies.

\section{Directory Structure}

Simple executables are in \$MCFBIN.  
The source code is in \$EVGEN\_DIR/src and 
\newline \$EVGEN\_DIR/inc.  
Input files are in \$EVGEN\_DIR/input.  
A scratch area is availble for large event files in 
/afs/fnal/files/scratch/bphysN, where N = 1-5.

The products cern, lund, herwig, isajet, qq, and 
stdhep will need to be defined.  Products are defined with the command 
"setup $[-t|-d]$ $<$product$> [version]$".  
The flags -t and -d signify test and development versions.  
Use the command "ups list -a $<$product$>$" to check the status of a product.

See http://www-pat.fnal.gov/stdhep.html for a table of stdhep and
related product versions.  Because common blocks and particle numbers
can change between event generator releases, it is important to
use the correct set of products.

\vspace{0.2 in}

\scriptsize
\begin{tabular}{|llll|} \hline
\multicolumn{4}{|c|}{Evgen directory structure} \\  \hline
\$EVGEN\_DIR  &    &     &    	\\
  & README   &     &     (evgen overview) \\
  & GNUmakefile   &     &     (build evgen library)	\\
  & evgen\_arch   &     &     (build evgen library)	\\
  &  doc & & \\
  &    &  examples.doc  &  (short explanation) \\
  &    &  evgen.tex  &  (long explanation) \\
  &  example  &    &  (example area) \\
  &    &  b\_psiks  &  (B to psi Ks example) \\
  &    &  d\_kpi  &  (D to K pi example) \\
  &    &  min\_bias  &  (minimum bias example) \\
  &    &  simple  &  (simple example) \\
  &  inc & & \\
  &    &  file\_names.inc & \\
  &    &  herwig\_cmd.inc & \\
  &    &  pythia\_user.inc & \\
  &    &  qq\_flag.inc & \\
  &  input & & \\
  &    &  isa\_nobc\_decay.dat  &  (isajet decay file if QQ used) \\
  &    &  jetset.dcy     &  (Pythia decay table if QQ used) \\
  &    &  user.dec     &  (sample user decay file for QQ) \\
  &  src & & \\
  &    &  amass\_isa.F & \\
  &    &  bookhistos.F & \\
  &    &  herwig\_exam.F & \\
  &    &  hwg\_command.F & \\
  &    &  hwg\_open.F & \\
  &    &  hwg\_user\_init.F & \\
  &    &  hwg\_user\_reset.F & \\
  &    &  isajet\_exam.F & \\
  &    &  main\_args.c & \\
  &    &  main\_fortran.F & \\
  &    &  pyt\_command.F & \\
  &    &  pyt\_get\_seed.F & \\
  &    &  pyt\_save\_seed.F & \\
  &    &  pyt\_user\_init.F & \\
  &    &  pythia\_exam.F & \\
  &    &  read\_xdr.F & \\
  &    &  usr\_end\_event.F & \\
  &    &  usr\_end\_job.F & \\
  &    &  usr\_filter.F & \\
  &    &  usr\_filter\_postqq.F & \\
  &    &  usr\_filter\_preqq.F & \\
\$MCFBIN  &    &     &    	\\
  &     herwig\_exam  & &  (generate herwig events) \\
  &     isajet\_exam  & &  (generate isajet events) \\
  &     pythia\_exam  & &  (generate pythia events) \\
  &     read\_xdr  & &  (example to read event file) \\
\$MCFLIB  &    &     &    	\\
  &  libevgen.a  &    &   \\
  & amass\_isa.o  &    &   \\
  & herwig\_exam.o  &    &   \\
  & isajet\_exam.o  &    &   \\
  & pythia\_exam.o  &    &   \\
  & read\_xdr.o  &    &   \\ \hline
\end{tabular}
\normalsize

\newpage

\section{Conventions}

We use the following file extension conventions.

\vspace{0.2 in}

\begin{tabular}{|l|l|} \hline
\multicolumn{2}{|c|}{File Extension Conventions} \\ \hline
 .hwg  & Herwig input file \\ \hline
 .isa  & Isajet input file \\ \hline
 .pyt  & Pythia input file \\ \hline
 .std  & input file for read\_xdr example \\ \hline
 .lpt  & line printer output file \\ \hline
 .evt  & xdr event files written by stdhep and mcfio \\ \hline
 .f  & standard fortran \\ \hline
 .F  & fortran which will be passed through the c preprocessor \\ \hline
 .rz  & hbook histogram or ntuple file \\ \hline
\end{tabular}

\section{Running the Executables}

Before running the examples, 
define the environmental variables QQ\_USER\_FILE, 
and ISAJET\_DECAY. 
To run an executable, type, for instance, \$MCFBIN/pythia\_exam.
You may define the input and output file names in the command line.
Alternatively, you may use the input file to define the file names.
For instance, you might type 
"\$MCFBIN/pythia\_exam -f pythia.pyt -o pythia\_1.evt -l pythia\_1.lpt"
Note that the structure of the Isajet input file is defined by Isajet.
The Herwig and Pythia input files are defined locally.
Sample input files are available in \$EVGEN\_DIR/example.
 

\vspace{0.2 in}

\begin{tabular}{|l|l|l|} \hline
\multicolumn{3}{|c|}{Command Line Arguments} \\ \hline
 -f & cmd\_file & input command file \\ \hline
 -l & lpt\_file & line printer output file \\ \hline
 -o & event\_output\_file & output event file \\ \hline
 -i & event\_input\_file & input event file \\ \hline
 -hb & hbook\_file & output hbook histogram file \\ \hline
 -qq &  & use QQ for decay \\ \hline
 -qu & user\_decay\_file & QQ user decay file \\ \hline
 -h  &  & type help display \\ \hline
\end{tabular}

\section{Sample Input files}

\begin{tabular}{|lll|} \hline
\multicolumn{3}{|c|}{Sample Herwig input file} \\  \hline
\multicolumn{3}{|l|}{!B production: 1800 GeV pbar on p} \\
file\_lpt & herwig\_exam.lpt &  !Output file \\
file\_evt\_wrt & herwig\_exam.evt &  !Output event file \\
file\_hbk & herwig\_exam.rz &  !hbook output \\
maxev & 1000      &          !Max events to generate \\
maxpr & 2        &           !Max number of events to print \\
ranseed & 0 0    &           !Random seed set to time of day \\
iproc & 1705    &            !bbar quark production \\
part1 & P      &             !Particle 1 ==$>$ Proton \\
part2 & PBAR   &             !Particle 2 ==$>$ Antiproton \\
pbeam1 & 900.    &           !Beam energy for proton \\
pbeam2 & 900. &   \\
ptmin & 0.        &          !Pt range \\
ptmax & 900. &  \\ \hline
\multicolumn{3}{|c|}{other possible commands} \\  \hline
maxer        &                 & ! \\
iprint       &                 & ! \\
yjmin        &                 & ! \\
yjmax        &                 & ! \\
q2min        &                 & ! \\
q2max        &                 & ! \\
emmin        &                 & ! \\
emmax        &                 & ! \\
qcdlam       &                 & ! \\
rmass        &                 & ! \\
file\_evt\_rd  &               & !input event file \\
qq           &                 & !use QQ for hadron decay \\
ranseed      &                 & !two numbers \\
exit         &                 & !stop reading the input file  \\
stop         &                 & !stop reading the input file  \\ \hline
\end{tabular}

\vspace{0.2 in}

\begin{tabular}{|lll|} \hline
\multicolumn{3}{|c|}{Sample Pythia input file} \\  \hline
\multicolumn{3}{|l|}{!B production: 900 GeV pbar on p} \\
file\_lpt     & pythia\_test.lpt & !Output print file \\
file\_evt\_wrt & pythia\_test.evt & !File to write events \\
file\_hbk     & pythia\_test.rz  & !Hbook output \\
maxev        & 1000            & !\# of events to generate \\
frame        & CMS             & !Select center of mass frame \\
beam         & PBAR            & !Beam particle 1 \\
target       & P               & !Beam particle 2 \\
cms\_energy   & 1800.0          & !Center of mass energy \\
ranseed      & 0               & !Set random seed using time of day \\
pygive       & MSEL=5          & !B production with massive matrix elements \\ \hline
\multicolumn{3}{|c|}{other possible commands} \\  \hline
maxpr        & 1               & !\# of events to print \\
file\_evt\_rd  & pythia\_prev.evt & !input event file \\
qq           &                 & !use QQ for hadron decay \\
exit         &                 & !stop reading the input file  \\
stop         &                 & !stop reading the input file  \\ \hline
\end{tabular}

\vspace{0.2 in}

\begin{tabular}{|l|} \hline
\multicolumn{1}{|c|}{Sample Isajet input file} \\  \hline
Example b bbar job \\
1800.,1000,2,10/ \\
TWOJET \\
BEAMS \\
'P','AP'/ \\
JETTYPE1 \\
'BT', 'BB'/ \\
JETTYPE2 \\
'BT', 'BB'/ \\
P \\
0. , 900., 0. , 900./ \\
PT \\
0.01 , 850., 0.01 , 850./ \\
NTRIES \\
5000/ \\
NSIGMA \\
400/ \\
END \\
STOP \\ \hline
See the Isajet manual for an explanation of options \\ \hline
\end{tabular}

\section{StdHep}

The Particle Data Group has established a standard particle numbering scheme.
A standard event common block, HEPEVT, has also been established.
The StdHep product is designed to convert various Monte Carlo event
generator output events to a standard version of the output event 
common block. 

The particle ID numbers between 1 and 80 are for elementary particles:
quarks, gluons, leptons, gauge and higgs bosons and their supersymmetric
partners. 
The PDG numbering algorithm for composite particles uses a 
signed 7 digit number for each particle:  $\pm nn_rn_Ln_{q_1}n_{q_2}n_{q_3}n_J$.
$n_{q_{1-3}}$ are quark numbers used to specify the quark content.
The rightmost digit, $n_J$ = 2J + 1, gives the spin of the composite particle.
The scheme does not cover particles of spin $J>4$.
The fifth digit, $n_L$, is reserved to distinguish mesons of the
same total ($J$) but different spin ($S$) and orbital ($L$)
angular momentum quantum numbers.
The sixth digit, $n_r$, is used to label mesons radially excited
above the ground state.
The numbering scheme does not extend to baryons with $n>0$, $n_r>0$, or $n_L>0$.
Digits $n_{q_2}$ and $n_{q_3}$ are used for mesons, with $n_{q_1}$ = 0.
Digits $n_{q_1}$, $n_{q_2}$, and $n_{q_3}$ are used for baryons.
Digits $n_{q_1}$ and $n_{q_2}$ are used for diquarks, with $n_{q_3}$ = 0. 
A negative number
indicates an antiparticle.  The states are generally listed in order of
increasing mass.  $K_L^0$ and $K_S^0$ are exceptions.  Their assigned
identification numbers are 130 and 310, respectively.
A complete explanation can be found in \$STDHEP\_DIR/doc/PDG\_numbers\_98.ps
or the 1998 PDG
( C. Caso et. al., The European Physical Journal {\bf C3} ).



\vspace{0.2 in}

\begin{tabular}{|ll|} \hline
\multicolumn{2}{|c|}{HEPEVT common block} \\  \hline
PARAMETER & (NMXHEP=4000) \\
COMMON/HEPEVT/ & NEVHEP,NHEP,ISTHEP(NMXHEP),IDHEP(NMXHEP), \\
   &  JMOHEP(2,NMXHEP),JDAHEP(2,NMXHEP), \\
 & PHEP(5,NMXHEP),VHEP(4,NMXHEP) \\ \hline
NEVHEP & event number \\
NHEP        & number of entries in this event record \\
ISTHEP(..)  & status code \\
IDHEP(..)   & particle ID, P.D.G. standard \\
JMOHEP(1,..)& position of mother particle in list \\
JMOHEP(2,..)& position of second mother particle in list \\
JDAHEP(1,..)& position of first daughter in list \\
JDAHEP(2,..)& position of last daughter in list \\
PHEP(1,..)  & x momentum in GeV/c \\
PHEP(2,..)  & y momentum in GeV/c \\
PHEP(3,..)  & z momentum in GeV/c \\
PHEP(4,..)  & energy in GeV \\
PHEP(5,..)  & mass in GeV/c**2\\
VHEP(1,..)  & x vertex position in mm \\
VHEP(2,..)  & y vertex position in mm \\
VHEP(3,..)  & z vertex position in mm \\
VHEP(4,..)  & production time in mm/c \\ \hline
\end{tabular}

\vspace{0.3 in}

\begin{tabular}{|ll|} \hline
\multicolumn{2}{|c|}{ISTHEP convention} \\  \hline
 0       & null\\
 1       & final state particle\\
 2       & intermediate state\\
 3       & documentation line\\
 4-10    & reserved for future use\\
 11-200  & reserved for specific model use\\
 201-... & reserved for users\\ \hline
\end{tabular}

\vspace{0.2 in}

Event I/O is handled by the routines STDXWRT(ILBL,ISTR,LOK) and 
STDXRD(ILBL,ISTR,LOK). 
LOK is an integer variable which returns the status of the I/O attempt.
ILBL is an integer which defines the type of record to read or write.
ISTR is a label used by MCFio to identify the event file.
STDXRD and STDXWRT are interfaces from StdHep to MCFio xdr binary files,
which can be transported to other platforms.  See the StdHep manual
for a more complete explanation.

\vspace{0.2 in}

\begin{tabular}{|ll|} \hline
\multicolumn{2}{|c|}{Supported Values of ILBL} \\  \hline
ILBL = 1    &  standard HEPEVT common block \\
ILBL = 2    &  standard HEPEVT common block plus multiple interaction info \\
ILBL = 100  &  stdhep begin run record \\
ILBL = 200  &  stdhep end run record \\
ILBL = anything else  &  use I/O buffer \\ \hline
\end{tabular}

\vspace{0.2 in}

StdHep writes the information in common block STDCM1 as the begin and
end run records.
You must call STDFLHWXSEC, STDFLISXSEC, or STDFLPYXSEC(N1) to fill
common block STDCM1 before calling HEPWRT with ILBL of 100 or 200.  
N1 is the number of events to be generated.


\vspace{0.2 in}

\begin{tabular}{|ll|} \hline
\multicolumn{2}{|c|}{STDCM1 Common Block} \\  \hline
      COMMON /STDCM1/ & STDECOM,STDXSEC,STDSEED1,STDSEED2, \\
     1            &   NEVTREQ,NEVTGEN,NEVTWRT \\ \hline
   STDECOM  & center-of-mass energy \\
   STDXSEC  & cross-section \\
   STDSEED1 & random number seed \\
   STDSEED2 & random number seed \\
   NEVTREQ  & number of events to be generated \\
   NEVTGEN  & number of events actually generated \\
   NEVTWRT  & number of events written to output file \\ \hline
\end{tabular}

\vspace{0.2 in}

The routine STDXRDM(ILBL,ISTR,LOK) is provided to allow users the option
of reading events from multiple input streams.  
STDXRDM adds each event to the end of the existing
data in HEPEVT.  The user must call STDZERO to zero the arrays in HEPEVT.
This is intended as a tool to fake multiple interactions.

See \$STDHEP\_DIR/doc/stdhep.ps for a list of stdhep translation,
I/O, and utility routines.

\end{document}


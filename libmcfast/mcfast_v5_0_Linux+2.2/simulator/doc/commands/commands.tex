\documentstyle[12pt,epsf]{article}


%\pagestyle{empty}
\parindent 0pt
\parskip 5mm

% declarations for front matter

\begin{document}
\begin{titlepage}




\begin{center}
{\LARGE \sc MCFast Commands}\\
\end{center}

\begin{abstract}
This document contains a description of MCFast commands and command files.    
\\

\end{abstract}
\begin{center}
MCFast v2\_5\_1 and v2\_6 and above \\
\end{center}
\end{titlepage}

\section{Running MCFast}

The fast Monte Carlo ``MCFast''is executed by typing

\centerline{         mcfast [-f command\_file] [-l lpt\_file] [-hb Hbook\_file]}

where 
 
\centerline{        command\_file = name of file containing job directives}
\centerline{        lpt\_file     = name of ASCII output file}
\centerline{        Hbook\_file   = name of output Hbook file}

Note: if the command file is not specified, mcfast accepts
directives from the terminal, terminated by a $\wedge$D (end of file).
This is very convenient when running multiple jobs in batch.

You can get online help on running mcfast by typing

\centerline{        mcfast -h}

with no other arguments.

During execution, mcfast checks the user's terminal every
event to see if anything has been typed. This feature can
be disabled for batch jobs by entering the line

\centerline{  batch ON}

in the command file.  Commands can be
abbreviated to the minimum necessary for uniqueness. Type
``help'' during execution to get a list of commands. Note
that if you type ``user'', the routine usr\_command is
called. You can provide this routine to print out any
information you want while the program is running.  

\filbreak
 List of commands that MCFast recognizes during execution(can be abbreviated):

\begin{description}
\item[ continue ]        Continue after a pause
\item[ exit/quit/stop ]  Stop reading data and finish
\item[ help   ]          Print this message
\item[ hcdir $<dir>$ ]     Change Hbook directory
\item[ hldir  ]          List contents of current Hbook directory
\item[ htype $<\#>$ ]       Type Hbook histogram in current directory
\item[ next ]            Analyze next event and pause
\item[ pause ]           Pause and wait for further commands
\item[ skip $\rm <n>$ ]        Read in but do not analyze the next n events
\item[ status  ]         Print a brief status message to terminal
\item[ user  ]           Call ``usr\_command'' (supplied by user)
\end{description}

\filbreak

\section{MCFast Command Files}

The ommand file format is as follows:

  input file format:\\
            keyword $value$\\
        or  keyword $value1\ \ value2$\\

    where keyword is one of the variables that can be set.

A whole series of reals, integers and strings can be read in. See the
routine {\it cmd\_getitems} for more details.

A sample command file for the C0 detector follows:



! C0 command file -- no transverse shower development, no xdr output\\
!\\
! use a ! to start a comment\\
!\\
!!!!!!!!!!!!!!!!!!!!!!!!!!!!!!!!!!!!!!!!!!!!!!!!!!!!!!!!!!!!!!!!!!!!!!!!!!!!!!!!\\
max\_ev 20000           ! Max number of events to analyze\\
max\_print 2            ! max \# events to print in lpt file\\
file\_geo c0.db         ! Geometry file\\
ranseed  0              ! Random seeds-- if 0 then use time of day\\
file\_type 1            ! file\_type 1 is the default from evgen\\
file\_in @psiks.fil     ! @filename is the file containing a list of data files\\
\\
!Example of simple trigger control\\
!\\
trigger\\
l1 define 0 track\_trigger\_1 use\\
l1 term track require num 1 pt 0.25 eta 1.5 5.0\\
\\
l1 define 1 track\_trigger\_2 use\\
l1 term track require num 1 pt 0.25 eta -5.0 -1.5\\
\\
l1 define 8 muon\_trig\_1 use\\
l1 term muon require num 1 pt 0.25 eta 1.5 5.0\\
\\
l1 define 9 muon\_trig\_2 use\\
l1 term muon require num 1 pt 0.25 eta -5.0 -1.5\\
end trigger\\
\\
\\
! program control   \\  
! use TRUE,  .TRUE.,  T, ON,  YES, Y\\
! or  FALSE, .FALSE., F, OFF, NO,  N\\

control\_shower\\
  shw\_trans    OFF   ! DO NOT generate shw.trans.prof. as shw propagates \\ 
  e\_thresh     0.05  ! \% of remaining energy to stop tracing shw. -- default  \\ 
  e\_vis\_min    0.002 ! minimum energy to read out in CAL cell  -- default  \\    
  max\_hit\_step   25  ! max \# of hits to generate per step  -- default   \\       
end control\_shower \\ 
 \\ 
 \\ 
! The following 3 quantities are set True by default in MCFast \\ 
make\_decays       T    ! do decays in flight \\ 
make\_pair\_convert T    ! do pair conversions \\ 
make\_hits     T        ! do generate hits \\ 


\filbreak
\section{MCFast Command Definitions}
The following terms are recognized by MCFast and can be entered in the command file.
\begin{description}
\item[max\_event]  -- Maximum number of event to be processed
\item[max\_print]  -- Maximum number of event to be printed in lpt file
\item[batch]      -- set ON for batch jobs; default = OFF
\item[file\_in]   @{\it filename} -- file containing list of signal files
\item[background\_in]@{\it filename} -- file containing list of background
\item[file\_out]  {\it filename}  -- output events in OLD STDHEP format-- obsolete
\item[file\_type] 1 --  input signal data file format; default = 1
\item[background\_type] -- input background data file format; default = 1
\item[ranseed]  --  enter 2 seeds for random number generator (RANECU); default = 0.0 0.0 (
if first seed is zero then MCFast uses time of day for seeds)
\item[int\_crossing] --  \# of interactions per crossing
\item[generate\_internal] --  initiate particle gun
\item[fcon] $<\#>$  {\it value} -- User defined floating point constant (500 allowed)
\item[icon] $<\#>$  {\it value} -- User defined integer constant (500 allowed)
\item[file\_geometry] {\it filename.db} -- geometry file name 
\item[luminosity] --     default = $10^{32}$ 
\item[runtime] -- default = $10^{7}$
\item[trigger] -- initiate trigger definitions
\item[dbg\_trace] -- include debug statements in MCFast; default = OFF -- see section on
DEBUG commands below
\item[dbg\_hist]  -- include debug histograms in MCFast; default = OFF -- see section on
DEBUG commands below
\item[qq\_user\_file] -- change QQ user file for decays in flight; default =
\$MCFAST\_DIR/examples/mcfast/stable.dec   
\item[make\_decays]  -- control flag for decays in flight;  default = ON
\item[make\_pair\_convert] -- control flag for pair conversions;  default = ON
\item[make\_hits]       -- control flag for hit generation; default = ON
\item[use\_mult\_scat]   -- contral flag for multiple scattering in trace
(v2\_6); default = OFF
\item[use\_energy\_loss] -- contral flag for  dedx in trace (v2\_6) default = OFF
\item[control\_shower] -- initiate shower control parameters
\item[step\_max\_distance]  -- maximum distance between trace points (for
graphics) default = 1.e10
\item[step\_max\_angle] --  maximum angle between trace points (for graphics)1.e10
\item[step\_max\_dedx]  -- maximum step for dedx (v2\_6) default = 1.e10
\item[step\_max\_mscat] -- maximum step for scattering (v2\_6) default = 1.e10
\item[trk\_max\_turns] -- maximum number of turns per step -- default = 0.5
\item[trk\_min\_kinetic] -- minimum kinetic energy for track -- default = 1.e6 MeV
\item[mcfio\_out]  -- mcfio output file name
\item[geomview\_out] -- geomview output file name -- this is obsolete.
\end{description}


\filbreak

\section{Particle Gun Command Definitions}
\begin{description}
\item[multiplicity] $<\#>$--  average multiplicity; default = 1
\item[fix\_multiplicity] ON -- default; if OFF then use Poisson distribution 
\item[slope] $slope$ --  default = 0.;    
if $slope\ \neq\ 0$ then $   p_t = ptmin - log(1 - random \# \times (1-exp(-slope \times (ptmax-ptmin)) )) / slope $
\item[type\_info] $hep\_id\ probability\ mass$ -- particle information; default = [211 1.0 0.13956]
\item[ptmin]  $ptmin$ --   minimum $p_t$ ; default = 1.
\item[ptmax]  $ptmax$ --   maximum $p_t$ ; default = 1.
\item[etamin] $etamin$ --  minimum eta; default = 0.
\item[etamax] $etamax$ --  maximum eta; default = 0.
\item[phimin] $phimin$ --  minimum phi angle; default = 0.
\item[phimax] $phimax$ --  maximum phi angle; default = $2 \pi$.
\item[vertex] $x\ y\ z$ -- primary vertex position; default = [0.0 0.0 0.0]
\item[dvertex] $dx\ dy\ dz$ --sigma of primary vertex position; default = [0.0 0.0 0.0]
\end{description}

\filbreak

\section{Debug Flags}

Used with command {\bf dbg\_trace} and {\bf dbg\_hist}:  

\centerline {{\bf dbg\_trace muon ON} ;  Turn on debug output for muon code}
\centerline {{\bf dbg\_hist muon ON}  ;  Turn on debug histograms in muon code}

The default is OFF for all subsystems.

BEWARE! Some of these switches turn on lots of output.  Many of them do nothing.

\begin{description}
\item[  general] ON sets dbg\_trace(jdbg\_general) .true. or dbg\_hist(jdbg\_general) 
\item[  track]   ON sets dbg\_trace(jdbg\_track) .true. or dbg\_hist(jdbg\_track)
\item[  shower] ON sets dbg\_trace(jdbg\_shower) .true. or dbg\_hist(jdbg\_shower)
\item[  gamma] ON sets dbg\_trace(jdbg\_gamma) .true. or dbg\_hist(jdbg\_gamma)
%\item[  hcal]  ON sets dbg\_trace(jdbg\_hcal) .true. or dbg\_hist(jdbg\_hcal)
\item[  emcal] ON sets dbg\_trace(jdbg\_emcal) .true. or dbg\_hist(jdbg\_emcal)
\item[  muon]  ON sets dbg\_trace(jdbg\_muon) .true. or dbg\_hist(jdbg\_muon)
\item[  trig]  ON sets dbg\_trace(jdbg\_trig) .true. or dbg\_hist(jdbg\_trig)
\item[  geom]  ON sets dbg\_trace(jdbg\_geom) .true. or dbg\_hist(jdbg\_geom)
\item[  hits]  ON sets dbg\_trace(jdbg\_hits) .true. or dbg\_hist(jdbg\_hits)
\end{description}

\filbreak

\section{Shower Control Command Definitions}
\begin{description}
\item[shw\_trans] -- 
     Default value is ON.  The transverse profile of showers are generated by default.
      In order to speed up the program, it is possible to create
     calorimeter hits only on the exact path of the shower core.   
\item[e\_thresh] -- 
      The default value is 0.05.  With the current showering algorithm, we stop tracing a
      shower if the remaining energy of the shower is 5\% or less of the incident energy.     
\item[max\_hit\_step] --  
The default value is 25.  This is the maximum number of
hits with status SHOWERING that can be produced over each step as a shower
propagates.  Correlated with {\bf e\_vis\_min} (nhits =
min((edep/e\_vis\_min),max\_hit\_step));  max\_hit\_step = 0 is equivalent to
shw\_trans OFF.  The user may change this parameter in order to speed up the code,
but the precision of the shower transverse profile generation may be affected.
\item[e\_vis\_min] -- 
The default value is 0.002 GeV.  This is the minimum energy
that can be recorded in the calorimeter cell (currently it is set once for all
calorimeters.) 
\end{description}

\filbreak

\section{Trigger Command Definitions}


Three types of triggers can be defined at a single level in 
the MCFAST framework.  The trigger types are TRACKING, MUON, and EM cluster.
These objects work only from the generator information.

To define triggers, several pieces of information must be specified.
\begin{enumerate}
 \item The level of the trigger (only L1 is active in this release)
 \item A key word corresponding to the function required 
 \begin{description}
   \item[define] -- sets up a specific trigger
   \item[term]   -- sets up the conditions corresponding to that specific trigger
 \end{description}
 
For ``define'' programming:

 \begin{description}
   \item[a)] bit number-- currently 0-31
   \item[b)] Name for the trigger
   \item[c)] Flag to determine whether or not to cut on the trigger condition.
 \end{description}
\end{enumerate}

Examples:

{\bf l1 define 2 track1\_trigger use}

\begin{description}

\item[ l1] --Trigger level
\item[ define ] --Key word
\item[ 2 ] --specific bit number (0-31)
\item[ track1\_trigger] --trigger name--should be $<$32 characters
\item[ use] -- cut on trigger condition, if left off, 
accounting for trigger will still appear
in summary, but trigger p/f will not be used.
\end{description}

Note that at this point, the trigger doesn't know anything about what
is required.  Ideally, the name should have some meaning to the user, but
what is used in the name has no effect on the trigger requirements.

For ``term'' programming:
\begin{description}
 \item[a)] A key word corresponding to the type of trigger requested
 \begin{description}
  \item[ muon]
  \item[ track]
  \item[ calem]
 \end{description}
 \item[b)] require or veto on this condition
 \item[c)] number of objects
 \item[d)] pt of object(s)
 \item[e)] eta of object(s)  --  For a symmetric geometry, be sure to specify 
                         - eta to + eta.
 \item[f)] phi of object(s)  -- if not used, a 2$\pi$ default is taken. 
\end{description}

Example:

{\bf l1 term track require num 1 pt 1.5 eta -10. 10.}

\begin{description}
\item[ l1 ] -- trigger level
\item[ term ] -- key word
\item[ track ] -- object type (track, muon or calem)
\item[ require] -- accept if the condition is true, use {\bf veto} to accept if the condition is false
\item[ num 1 ] -- multiplicity
\item[ pt 1.5 ] -- pt threshold
\item[ eta -10. 10. ] -- eta range
\end{description}

This supplies the requirements to the trigger.  There can be up to 256
terms associated with each specific trigger.  Examples of complete 
triggers are:

l1 define 0 test \\
l1 term track require num 1 pt 0 eta -10. 10.\\
{just count the events with any tracks at all}\\

l1 define 1 track1\_trigger use\\
l1 term track require num 1 pt 1.5 eta 0. 10.\\

l1 define 2 track2\_trigger use\\
l1 term track require num 1 pt 3 eta -10. 10.\\
l1 term track require num 2 pt 1.5 eta -10. 10.\\

l1 define 3 muon\_trig use\\
l1 term muon require num 1 pt 2.0 eta 1. 4.\\
l1 term track  require num 1 pt 2.0 eta 1. 4.\\
{note, the muon and track are not  ``matched''}\\

Output:  At the end of the mcfast.lpt file, trigger summary is printed out
showing a truncated version of the term definitions, the number of events
tried and passed for each term and each trigger.  Table \ref{tab:l1terms} shows
the output from a typical job with 2 track triggers, 1 muon trigger and 4 
calorimetry triggers defined.

\begin{table*}
% space before first and after last column: 1.5pc
% space between columns: 3.0pc (twice the above)
%\setlength{\tabcolsep}{1.5pc}
% -----------------------------------------------------
% adapted from TeX book, p. 241
%\newlength{\digitwidth} \settowidth{\digitwidth}{\rm 0}
%\catcode`?=\active \def?{\kern\digitwidth}
% -----------------------------------------------------
\caption{  Trigger Programming Summary:  level 1 term definitions}
\label{tab:l1terms}
\begin{tabular*}{\textwidth}{@{}l@{\extracolsep{\fill}}crr}
\hline
    TERM  &          TERM NAME              &            events passed /& tried \\
\hline 
       1  &   l1 term track require num 1 pt 1.5 eta 0      &    347/   &  500\\
       2  &   l1 term track require num 1 pt 1.5 eta -      &    485/   &  500\\
      33  &   l1 term muon require num 1 pt 2.0 eta 0.      &    168/   &  500\\
      65  &   l1 term calem require num 1 pt 0.0 eta -      &    335/   &  500\\
      66  &   l1 term calem require num 1 pt 0.0 eta 0      &    221/   &  500\\
      67  &   l1 term calem require num 1 pt 0.1 eta -      &     49/   &  500\\
      68  &   l1 term calem require num 1 pt 0.1 eta 0      &     23/   &  500\\
\hline
\end{tabular*}
\end{table*}

\filbreak
 LEVEL 1 specific triggers \\
 Trigger 0  TRACK\_TRIGGER                   required           347/     500 \\
    terms used:  1\\
 Trigger 1  TRACK1\_TRIGGER                  required           485/     500  \\
    terms used:  2 \\



\end{document}



